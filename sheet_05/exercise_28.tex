\section{}

The main idea of our approach is taken from \cite[6.7]{noncommutative_noetherian}.

Let~$A = \bigcup_i A_{(i)}$ be a filtered algebra (where~$A_{(-j)} = 0$ for all~$j \leq -1$).
We denote for~$x \in A_{(i)}$ the corresponding equivalence class in~$\gr[i](A) = A_{(i)} / A_{(i-1)}$ by~$[x]_i$.
We denote for every~$a \in A$ by~$\deg(a)$ the degree of~$a$, i.e.
\[
  \deg(a)
  =
  \min
  \{
    i \geq 0
  \suchthat
    a \in A_{(i)}
  \} \,.
\]
For every element~$a \in A$ the equivalence class~$[a]_i$ is defined for all~$i \geq \deg(a)$, but~$[a]_i = 0$ if~$i > \deg(a)$.
To every element~$a \in A$ we can hence associate a canonical element~$\gamma(a) \in \gr(A)$, namely~$[a]_{\deg(a)}$.

To every left (resp.\ right) ideal~$I$ in~$A$ we can associate a homogeneous left (resp.\ right) ideal~$\gr(I)$ in~$A$, given by the homogeneous parts
\[
  \gr[i](I)
  \defined
  ( A_{(i-1)} + I \cap A_{(i)} ) / A_{(i-1)}
  =
  \{
    [x]_i
  \suchthat
    x \in I \cap A_{(i)}
  \}
  \subseteq
  \gr[i](A)
\]
If~$I$ is a left ideal then~$\gr(I) = \bigoplus_{i \geq 0} \gr[i](I)$ is a left ideal in~$\gr(A)$ because
\begin{align*}
  \gr[i](A) \gr[j](I)
  &=
  \{
    [a]_i [x]_j
  \suchthat{}
    [a]_i \in \gr[i](A),
    [x]_j \in \gr[j](I)
  \}
  \\
  &=
  \{
    [a]_i [x]_j
  \suchthat
    a \in A_{(i)},
    x \in I \cap A_{(j)}
  \}
  \\
  &=
  \{
    [a x]_{i+j}
  \suchthat
    a \in A_{(i)},
    x \in I \cap A_{(j)}
  \}
  \\
  &\subseteq
  \{
    [y]_{i+j}
  \suchthat
    y \in I \cap A_{(i+j)}
  \}
  \\
  &=
  \gr[i+j](I) \,.
\end{align*}
If~$I$ is a right ideal then a similar calculation shows that~$\gr(I)$ is indeed a right ideal in~$\gr(A)$.

\begin{recall}
  If~$M$ is an~{\module{$R$}} (where~$R$ is some ring) and~$A, B, C$ are submodules of~$M$ with~$A \subseteq C$ then
  \[
    A + (B \cap C) = (A + B) \cap C \,.
  \]
  We can therefore just write~$A + B \cap C$.
\end{recall}

\begin{remark}
  The ideal~$I$ inherits from~$A$ a filtration given by~$I_{(i)} \defined I \cap A_{(i)}$.
  Then by the second isomorphism theorem
  \begin{align*}
    \gr[i](I)
    &=
    (A_{(i-1)} + I \cap A_{(i)}) / A_{(i-1)}
    \\
    &\cong
    (I \cap A_{(i)}) / (A_{(i-1)} \cap I \cap A_{(i)})
    \\
    &=
    (I \cap A_{(i)}) / (I \cap A_{(i-1)})
    \\
    &=
    I_{(i)} / I_{(i-1)} \,.
  \end{align*}
  This justifies the notation~$\gr[i](I)$.
\end{remark}

If~$I$ and~$J$ are two ideals in~$A$ with~$I \subseteq J$ then also~$\gr[i](I) \subseteq \gr[i](J)$ for all~$i$ and thus~$\gr(I) \subseteq \gr(J)$.
We will see that if~$I$ is strictly contained in~$J$ then~$\gr(I)$ is also strictly contained in~$\gr(J)$.
This will be a consequence of the following observation, that is interesting in its own right.

\begin{proposition}
  \label{pulled back generating set}
  Let~$I$ be a left (resp.\ right) ideal in~$A$ and let~$S$ be a subset of~$I$.
  If the associated elements~$[s]_{\deg(s)}$ with~$s \in S$ is a generating set for the left (resp.\ right) ideal~$\gr(I)$ then~$S$ is a generating set for~$I$.
\end{proposition}

\begin{proof}
  We consider only the case that~$I$ is a left ideal;
  the case of a right ideal works the same.
  
  Let~$x \in I$.
  The associated element~$[x]_{\deg(x)} \in \gr(I)$ can be written as a linear combination
  \[
    [x]_{\deg(x)}
    =
    \sum_{s \in S} b_s [s]_{\deg(s)}
  \]
  for some~$b_s \in \gr(A)$.
  By decomposing the coefficients~$b_s$ into homogeneous components we see that we can replace each~$b_s$ by its homogeneous component of degree~$\deg(x) - \deg(s)$.
  We may hence assume that each~$b_s$ is homogeneous of degree~$\deg(x) - \deg(s)$.
  We can now write every coefficient~$b_s$ as~$b_s = [a_s]_{\deg(x) - \deg(s)}$ for some~$a_s \in A_{\deg(x) - \deg(s)}$.%
  \footnote{If~$\deg(s) > \deg(x)$ then~$b_s = 0$ and~$a_s = 0$.}
  Then
  \begin{align*}
    [x]_{\deg(x)}
    &=
    \sum_{s \in S} b_s [s]_{\deg(s)}
    \\
    &=
    \sum_{s \in S} [a_s]_{\deg(x) - \deg(s)} [s]_{\deg(s)}
    \\
    &=
    \sum_{s \in S} [a_s s]_{\deg(x)}
    \\
    &=
    \Biggl[ \sum_{s \in S} a_s s \Biggr]_{\deg(x)}
  \end{align*}
  and hence~$x - \sum_{s \in S'} a_s s \in A_{\deg(x) - 1}$.
  Now we can proceed inductively to express this difference as a linear combination of~$S$.
\end{proof}

\begin{remark}
  One may think for~$a \in A$ about the associated element~$[a]_{\deg(a)} \in \gr(A)$ as the \enquote{leading term} of~$a$, and for an ideal~$I$ in~$A$ about~$\gr(I)$ as the \enquote{ideal of leading terms} of~$I$.
  (Note that if~$A$ is a graded algebra then~$\gr(A) = A$ and~$[a]_{\deg(a)}$ is precisely the leading term of~$a$ in the usual sense.
  Then also~$\gr(I) = I$ for every homogeneous left (resp.\ right) ideal~$I$ in~$A$.)
  Then the above statement and its proof may be compared to Hilbert’s basis theorem%
  \footnote{In \cite[Proposition~6.7]{noncommutative_noetherian} the above \lcnamecref{pulled back generating set} is implicitely used but not explicitely stated.
  For the missing calculations the authors instead refer to an earlier proof --- namely that of Hilbert’s basis theorem.}%
  , or the concept of a Gröbner~basis.
\end{remark}

% TODO: Is I uniquely determined by gr(I)?

\begin{corollary}
  \label{strictness is preserved}
  If~$I$ and~$J$ are left (resp.\ right) ideals in~$A$ with~$I \subsetneq J$ then~$\gr(I) \subsetneq \gr(J)$.
\end{corollary}

\begin{proof}
  Suppose that~$\gr(I) = \gr(J)$.
  The ideal~$\gr(I)$ is homogeneous and thus generated by homogeneous elements.
  There hence exists some subset~$S$ of~$I$ such that~$\gr(I)$ is generated by the associated elements~$[s]_{\deg(s)}$ with~$s \in S$.
  (Here we use that the homogeneous elements contained in~$\gr(I)$ are precisely those of the form~$[x]_{\deg(x)}$ with~$x \in I$.)
  It follows from \cref{pulled back generating set} and~$\gr(I) = \gr(J)$ that~$S$ is a generating set of both~$I$ and~$J$.
  Thus~$I = J$.
\end{proof}

\begin{corollary}
  \label{associated graded reflects noetherian and artinian}
  If~$\gr(A)$ is left (resp.\ right) noetherian then~$A$ is left (resp.\ right) noetherian.
  The same holds for left (resp.\ right) artinian.
\end{corollary}

\begin{proof}
  Every strictly increasing sequence of left (resp.\ right) ideals
  \[
    I_0
    \subsetneq
    I_1
    \subsetneq
    I_2
    \subsetneq
    I_3
    \subsetneq
    \dotsb
  \]
  in~$A$ results by \cref{strictness is preserved} in a strictly increasing sequence of left (resp.\ right) ideals
  \[
    \gr(I_0)
    \subsetneq
    \gr(I_1)
    \subsetneq
    \gr(I_2)
    \subsetneq
    \gr(I_3)
    \subsetneq
    \dotsb
  \]
  in~$\gr(A)$.
  So if~$A$ is not noetherian then neither is~$\gr(A)$.
  Similarly for artinian.
\end{proof}

\begin{corollary}
  If~$\glie$ is a finite dimensional Lie~algebra then~$\Univ(\glie)$ is both left and right noetherian.
\end{corollary}

\begin{proof}
  If~$\glie$ is of dimension~$n$ then~$\gr(\Univ(\glie)) \cong k[x_1, \dotsc, x_n]$ is both left noetherian and right noetherian whence the assertion follows from \cref{associated graded reflects noetherian and artinian}.
\end{proof}

\begin{remark}
  Recall that for any~$n \geq 1$ the polynomial algebra~$k[x_1, \dotsc, x_n]$ is not artinian because
  \[
    (x_n)
    \supsetneq
    (x_n^2)
    \supsetneq
    (x_n^3)
    \supsetneq
    \dotsb
  \]
  is a strictly decreasing sequence of ideals of infinite length.
  We can see similarly that if~$\glie$ is any nonzero Lie~algebra then~$\Univ(\glie)$ is neither left artinian nor right artinian:
  
  We show that that~$\Univ(\glie)$ is not left artinian:
  Let~$(x_i)_{i \in I}$ be basis of~$\glie$ such that the index set~$(I, \leq)$ is linearly ordered and admits a maximal element~$j \in I$.
  (Here we use the well ordering theorem.)
  Then~$\Univ(\glie)$ admits the associated PBW basis consisting of all monomials~$x_{i_1}^{n_1} \dotsm x_{i_r}^{n_r}$ with~$r \geq 0$,~$n_i \geq 1$,~$i_1 < \dotsb < i_r$.
  It follows that for all~$m \geq 0$ the left ideal~$\Univ(\glie) x_j^m$ has as a basis all monomials~$x_{i_1}^{n_1} \dotsm x_{i_r}^{n_r} x_j^{m'}$ with~$r \geq 0$,~$n_i \geq 1$,~$i_1 < \dotsb < i_r < j$ and~$m' \geq m$.
  Thus
  \[
    \Univ(\glie) x_j
    \supsetneq
    \Univ(\glie) x_j^2
    \supsetneq
    \Univ(\glie) x_j^3
    \supsetneq
    \dotsb
  \]
  is a strictly increasing sequence of left ideals of infinite length, showing that~$\Univ(\glie)$ is not left artinian.
  
  That~$\Univ(\glie)$ is not right artinian can be seen in the same way.
\end{remark}


