\section{}

The relations
\[
  [h,e] = 2e \,,
  \qquad
  [h,f] = -2f \,,
  \qquad
  [e,f] = h
\]
can be regarded as \emph{rewritting rules}
\[
  hf \to fh - 2f \,,
  \qquad
  eh \to he - 2e \,,
  \qquad
  ef \to fe + h \,.
\]
To write a monomial~$e^l h^m f^n$ as a linear combination of the given basis monomials we repeatedly apply this rewritting rules and expand the resulting expressions if needed.
This is best done using computers.
We show two ways of doing this.





\subsection*{Python (Sympy)}

The first solution uses \texttt{python}.
We use the \texttt{sympy} (symbolic python) package, which allows us to use symbolic manipulation and replacement.

\begin{pythoncode}
from sympy import *

e, h, f = symbols('e h f', commutative=False)

def sl2expand(term):
  while True:
    newterm = term
    newterm = newterm.subs(h*f, f*h - 2*f)
    newterm = newterm.subs(e*h, h*e - 2*e)
    newterm = newterm.subs(e*f, f*e + h)
    newterm = expand(newterm)
    if newterm == term:
      break
    else:
      term = newterm
  return term
\end{pythoncode}
We get from this program the following results:
\begin{consoleoutput}
>>> sl2expand(e * h * f)
-4*f*e + f*h*e - 2*h + h**2
>>> sl2expand(e**2 * h**2 * f**2)
-288*f*e + 240*f*h*e - 56*f*h**2*e + 4*f*h**3*e + 64*f**2*e**2 - 16*f**2*h*e**2 + f**2*h**2*e**2 - 32*h + 48*h**2 - 18*h**3 + 2*h**4
>>> sl2expand(h * f**3)
-6*f**3 + f**3*h
>>> sl2expand(h**3 * f)
-8*f + 12*f*h - 6*f*h**2 + f*h**3
\end{consoleoutput}
Hence
\begin{align*}
  ehf
  ={}&
  - 4 f e + f h e - 2 h + h^2
  \\
  ={}&
  f h e - 4 f e + h^2 - 2 h \,,
  \\
  e^2 h^2 f^2
  ={}&
  -288 f e + 240 f h e - 56 f h^2 e + 4 f h^3 e + 64 f^2 e^2
  \\
  {}&
  - 16 f^2 h e^2 + f^2 h^2 e^2 - 32 h + 48 h^2 - 18 h^3 + 2 h^4
  \\
  ={}&
  f^2 h^2 e^2 - 16 f^2 h e^2 + 64 f^2 e^2 + 4 f h^3 e - 56 f h^2 e
  \\
  {}&
  + 240 f h e - 288 f e + 2 h^4 - 18 h^3 + 48 h^2 - 32 h \,,
  \\
  h f^3
  ={}&
  -6 f^3 + f^3 h
  \\
  ={}&
  f^3 h - 6 f^3 \,,
  \\
  h^3 f
  ={}&
  -8 f + 12 f h - 6 f h^2 + f  h^3
  \\
  ={}&
  f h^3 - 6 f h^2 + 12 f h - 8 f \,.
\end{align*}





\subsection*{Sage}

Another software solution is provided by \texttt{sage}, a computer algebra system that provides many tools for algebra, combinatorics, representation theory, etc.
Lie algebras and their PBW bases are already implemented.
An explicit explanation of PBW bases can be found in \cite{sage_pbw}.

% After a first compiling we get a *.sagetex.sage file, on which we have to run sage.
% Then we need to compile the document again.
\begin{sagecommandline}
sage: g = lie_algebras.three_dimensional_by_rank(QQ, 3, names=['E','F','H'])

sage: def sort_key(x):                                       
....:   if x == 'F':
....:     return 0
....:   if x == 'H':
....:     return 1
....:   if x == 'E':
....:     return 2


sage: pbw = g.pbw_basis(basis_key=sort_key)
sage: E,F,H = pbw.algebra_generators()

An easy test:
sage: H * E - E * H
sage: H * F - F * H
sage: E * F - F * E

Now the real deal:
sage: E * H * F
sage: E^2 * H^2 * F^2
sage: H * F^3
sage: H^3 * F
\end{sagecommandline}
(The above output in the lines 12,~14,~16,~18,~20,~22,~24 is in fact autogenerated by \texttt{sage} when compiling this document.)

\begin{remark}
  The term~$h f^3$ --- and more generally~$h f^n$ for every~$n \geq 0$ --- can also be computed smartly:
  If~$A$ is any~{\algebra{$k$}} and~$a \in A$ then the map~$[a,-] \colon A \to A$ is a derivation of~$A$, i.e.
  \[
    [a, xy]
    =
    [a,x] y + x [a,y] \,.
  \]
  In~$A = \Univ(\sllie_2(k))$ we therefore have
  \[
    [h, f^n]
    =
    [h, f f \dotsm f]
    =
      [h,f] f \dotsm f
    + f [h, f] \dotsm f
    + \dotsb
    + f f \dotsm [h,f] \,.
  \]
  With~$[h,f] = -2f$ we find
  \[
    [h, f^n]
    =
    -2 n f^n \,.
  \]
  With~$[h, f^n] = h f^n - f^n h$ we find by rearranging that
  \[
    h f^n = f^n h - 2 n f^n \,.
  \]
\end{remark}

