\section{}

\begin{lemma}
  \label{weight space of tensor product}
  For any two finite dimensional {\representations{$\sllie_2(k)$}}~$V$,~$W$ and every~$\kappa \in k$,
  \[
    (V \tensor W)_\kappa
    =
    \bigoplus_{\lambda + \mu = \kappa}
    V_\lambda \tensor W_\mu \,.
  \]
\end{lemma}

\begin{proof}
  We have~$V_\lambda \tensor W_\mu \subseteq (V \tensor W)_{\lambda + \mu}$ because
  \[
    h.(v \tensor w)
    =
    (h.v) \tensor w + v \tensor (h.w)
    =
    \lambda v \tensor w + \mu v \tensor w
    =
    (\lambda + \mu) v \tensor w
  \]
  for every simple tensor~$v \tensor w \in V_\lambda \tensor W_\mu$.
  We also know from Weyl’s~theorem that
  \begin{gather*}
    V \tensor W
    =
    \bigoplus_\kappa (V \tensor W)_\kappa
  \shortintertext{and}
    V \tensor W
    =
    \biggl( \bigoplus_\lambda V_\lambda \biggr)
    \tensor
    \biggl( \bigoplus_\mu W_\mu \biggr)
    =
    \bigoplus_{\lambda, \mu} V_\lambda \tensor W_\mu
    =
    \bigoplus_\kappa
    \bigoplus_{\lambda + \mu = \kappa} V_\lambda \tensor W_\mu \,.
  \end{gather*}
  We see that the inclusion~$\bigoplus_{\lambda + \mu = \kappa} V_\lambda \tensor W_\mu \subseteq (V \tensor W)_\kappa$ is already an equality.
\end{proof}

For every finite dimensional {\representation{$\sllie_2(k)$}}~$V$ its \emph{formal character} is
\[
  \ch(V)
  \defined
  \sum_{n \in \Integer} \dim(V_n) t^n
  \in
  k[t, t^{-1}] \,.
\]

\begin{lemma}
  \leavevmode
  \begin{enumerate}
    \item
      For every~$n \geq 0$,
      \[
        \ch(\irr(n))
        =
        t^n + t^{n-2} + \dotsb + t^{2-n} + t^{-n} \,.
      \]
    \item
      If~$V^{(1)}, \dotsc, V^{(n)}$ are finite dimensional {\representations{$\sllie_2(k)$}} then
      \[
        \ch\bigl( V^{(1)} \oplus \dotsb \oplus V^{(n)} \bigr)
        =
        \ch\bigl( V^{(1)} \bigr) + \dotsb + \ch\bigl( V^{(n)} \bigr) \,.
      \]
    \item
      If~$V$ and~$W$ are two finite dimensional {\representations{$\sllie_2(k)$}} then
      \[
        \ch(V \tensor W)
        =
        \ch(V) \ch(W) \,.
      \]
  \end{enumerate}
\end{lemma}

\begin{proof}
  \leavevmode
  \begin{enumerate}
    \item
      We see this from the explicit description of~$\irr(n)$ as done in the lecture.
    \item
      This follows from~$( V^{(1)} \oplus \dotsb \oplus V^{(n)} )_\lambda = V^{(1)}_\lambda \oplus \dotsb \oplus V^{(n)}_\lambda$.
    \item
      This follows from \cref{weight space of tensor product}.
    \qedhere
  \end{enumerate}
\end{proof}

\begin{proposition}
  Two finite dimensional {\representations{$\sllie_2(k)$}}~$V$ and~$W$ are isomorphic if and only if they have the same formal character.
\end{proposition}

\begin{proof}
  If~$V \cong W$ then~$\dim(V_\lambda) = \dim(W_\lambda)$ for every~$\lambda \in k$ and thus~$\ch(V) = \ch(W)$.
  
  Suppose now that~$\ch(V) = \ch(W)$.
  By Weyl’s~theorem and the classification of finite dimensional irreducible {\representations{$\sllie_2(k)$}} we can decompose both~$V$ and~$W$ as
  \begin{alignat*}{3}
    V
    \cong&
    \bigoplus_{n=0}^{N} \irr(n)^{\oplus p_n}
    &
    \quad
    &\text{and}
    \quad
    &
    W
    \cong&
    \bigoplus_{m=0}^{M} \irr(m)^{\oplus q_m}
  \intertext{
  with~$p_{N}, q_{M} \neq 0$.
  Then~$\ch(V)$ has the leading term~$p_N t^N$ and~$\ch(W)$ has the leading term~$q_M t^M$.
  It follows from~$\ch(V) = \ch(W)$ that~$N = M$ and~$p_N = q_M$.
  It follows for
  }
    V'
    \defined&
    \bigoplus_{n=0}^{N-1} \irr(n)^{\oplus p_n}
    &
    \quad
    &\text{and}
    \quad
    &
    W'
    \defined&
    \bigoplus_{m=0}^{M-1} \irr(m)^{\oplus q_m}
  \end{alignat*}
  that
  \[
    \ch(V')
    =
    \ch(V) - \ch\bigl( \irr(N)^{\oplus p_N} \bigr)
    =
    \ch(W) - \ch\bigl( \irr(M)^{\oplus q_M} \bigr)
    =
    \ch(W')
  \]
  We can now apply induction to find~$V' \cong W'$ and hence~$p_i = q_i$ for all~$i = 0, \dotsc, N-1$.
  The induction start is given by~$\ch(V) = \ch(W) = 0$, in which case~$V = 0 = W$.
  
  This shows that~$N = M$ and~$p_i = q_i$ for all~$i = 1, \dotsc, N$, whence~$V \cong W$.
\end{proof}

Let now~$n, m \geq 0$.
To determine the decomposition of~$\irr(n) \tensor \irr(m)$ into irreducibles we may assume that~$n \geq m$ (because~$\irr(n) \tensor \irr(m) \cong \irr(m) \tensor \irr(n)$).
We can now proceed in two ways:
\begin{itemize}
  \item
    We have
    \begin{align*}
      {}&
      \ch(\irr(n) \tensor \irr(m))
      \\
      ={}&
      \ch(\irr(n)) \ch(\irr(m))
      \\
      ={}&
      (t^n + t^{n-2} + \dotsb + t^{2-n} + t^{-n})
      (t^m + t^{m-2} + \dotsb + t^{2-m} + t^{-m})
      \\
      ={}&
          t^{n+m}
      + 2 t^{n+m-2}
      + \dotsb
      + m t^{n-m}
      + \dotsb
      + m t^{m-n}
      + \dotsb
      + 2 t^{2-n-m}
      + t^{-n-m} \,.
      \\
      ={}&
      (t^{n+m} + t^{n+m-2} + \dotsb + t^{2-n-m} + t^{-n-m})
      \\
      {}&
      + (t^{n+m-2} + t^{n+m-4} + \dotsb + t^{4-n-m} + t^{2-n-m})
      \\
      {}&
      + \dotsb
      \\
      {}&
      + (t^{n-m} + t^{n-m-2} + \dotsb + t^{m-n})
      \\
      ={}&
      \ch(\irr(n+m)) + \ch(\irr(n+m-2)) + \dotsb + \ch(\irr(n-m))
      \\
      ={}&
      \ch\bigl( \irr(n+m) \oplus \irr(n+m-2) \oplus \dotsb \oplus \irr(n-m) \bigr)
    \end{align*}
    and therefore
    \[
      \irr(n) \tensor \irr(m)
      \cong
      \irr(n+m) \oplus \irr(n+m-2) \oplus \dotsb \oplus \irr(n-m) \,.
    \]
  \item
    We note that
    \[
      \ch(\irr(n))
      =
      t^n + t^{n-2} + \dotsb + t^{2-n} + t^{-n}
      =
      \frac{t^{n+2} - t^{-n}}{t^2 - 1} \,.
    \]
    It follows that
    \begin{align*}
      {}&
      \ch(\irr(n) \tensor \irr(m))
      \\
      ={}&
      \ch(\irr(n)) \ch(\irr(m))
      \\
      ={}&
      \frac{t^{n+2} - t^{-n}}{t^2 - 1} (t^m + t^{m-2} + \dotsb + t^{2-m} + t^{-m})
      \\
      ={}&
      \frac{(t^{n+2} - t^{-n})(t^m + t^{m-2} + \dotsb + t^{2-m} + t^{-m})}{t^2 - 1}
      \\
      ={}&
      \frac{ t^{n+m+2} + t^{n+m} + \dotsb + t^{n-m+2} - t^{m-n} - \dotsb - t^{-n-m+2} - t^{-n-m} }{t^2 - 1}
      \\
      ={}&
        \frac{t^{n+m+2} - t^{-n-m}}{t^2 - 1}
      + \frac{t^{n+m} - t^{-n-m+2}}{t^2 - 1}
      + \dotsb
      + \frac{t^{n-m+2} - t^{m-n}}{t^2 - 1}
      \\
      ={}&
      \ch(\irr(n+m)) + \ch(\irr(n+m-2)) + \dotsb + \ch(\irr(n-m))
      \\
      ={}&
      \ch\bigl( \irr(n+m) \oplus \irr(n+m-2) \oplus \dotsb \oplus \irr(n-m) \bigr)
    \end{align*}
    and hence again
    \[
      \irr(n) \tensor \irr(m)
      \cong
      \irr(n+m) \oplus \irr(n+m-2) \oplus \dotsb \oplus \irr(n-m) \,.
    \]
\end{itemize}

The above decomposition of~$\irr(n) \tensor \irr(m)$ is known as the \emph{Clebsch--Gordan decomposition} or \emph{Clebsch--Gordan rule}.
Under the name of \emph{Glebsch--Gordan coefficients} this plays a role in quantum mechanics.




