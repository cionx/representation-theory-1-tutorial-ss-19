\section{}





\addtocounter{subsection}{1}





\addtocounter{subsection}{1}





\subsection{}

We follow the hint:
Let~$V$ be a {\fourdimensional} vector space with basis~$e_1, e_2, e_3, e_4$.
We identify the exterior power~$\exterior[4]{V}$ with the ground field~$\Complex$ via the single basis element~$e_1 \wedge e_2 \wedge e_3 \wedge e_4$.

We find that
\[
  v \wedge w_1 \wedge w_2
  =
  - w_1 \wedge v \wedge w_2
  =
  w_1 \wedge w_2 \wedge v
\]
for every vector~$v \in V$ and every simple wedge~$w_1 \wedge w_2 \in \exterior[2]{V}$.
The bilinear map
\[
  \beta
  \colon
  \exterior[2]{V}
  \times
  \exterior[2]{V}
  \to
  \exterior[4]{V} \,,
  \quad
  (x,y)
  \mapsto
  x \wedge y
\]
is hence symmetric, beause
\begin{align*}
  \beta(v_1 \wedge v_2, w_1 \wedge w_2)
  &=
  v_1 \wedge v_2 \wedge w_1 \wedge w_2
  \\
  &=
  v_1 \wedge w_1 \wedge w_2 \wedge v_2
  \\
  &=
  w_1 \wedge w_2 \wedge v_1 \wedge v_2
\end{align*}
for all simple wedges~$v_1 \wedge v_2, w_1 \wedge w_2 \in \exterior[2]{V}$.
The bilinear form~$\beta$ is also non-degenerate:
A basis of the exterior square~$\exterior[2]{V}$ is given by the simple wedges
\[
  e_1 \wedge e_2  \,,
  \quad
  e_1 \wedge e_3  \,,
  \quad
  e_1 \wedge e_4  \,,
  \quad
  e_2 \wedge e_3  \,,
  \quad
  e_2 \wedge e_4  \,,
  \quad
  e_3 \wedge e_4  \,.
\]
With respect to this basis of~$\exterior[2]{V}$ the bilinear form~$\beta$ is given by the matrix
\[
  \begin{pmatrix*}[r]
      &     &   &   &     & 1 \\
      &     &   &   & -1  &   \\
      &     &   & 1 &     &   \\
      &     & 1 &   &     &   \\
      & -1  &   &   &     &   \\
    1 &     &   &   &     &  
  \end{pmatrix*}  \,.
\]
This matrix is invertible, which means that~$\beta$ is non-degenerate.

\begin{warning}
  The fourth exterior power~$\exterior[4]{V}$ is neither the same nor isomorphic to the iterated exterior square~$\exterior[2]{\exterior[2]{V}}$.
  In particular~$\dim \exterior[4]{V} = \binom{\dim V}{4} = \binom{4}{4} = 1$ whereas~$\dim \exterior[2]{\exterior[2]{V}} = \binom{\dim \exterior[2]{V}}{2} = \binom{6}{2} = 15$.
\end{warning}

It follows from Problem~8 of the first exercise sheet that
\[
  \glie_\beta
  \defined
  \biggl\{
    \varphi
    \in
    \gllie(V)
    \suchthat[\bigg]
    \text{$\beta(\varphi(x), y) + \beta(x, \varphi(y)) = 0$ for all~$x,y \in \exterior[2]{V}$}
  \biggr\}
\]
is a Lie~subalgebra of~$\gllie(V)$ that is isomorphic to~$\solie_6(\Complex)$, because~$\beta$ in non-degenerate and symmetric and~$\exterior[2]{V}$ is~{\sixdimensional}.
We will in the following construct an explicit isomorphism~$\sllie_4(V) \cong \glie_\beta$ (where~$\sllie(V) \cong \sllie_4(\Complex)$ because~$V$ is~{\fourdimensional}).

\begin{lemma}
  \label{induced exterior action}
  If~$V$ is a representation of a Lie~algebra~$\glie$ then any exterior power~$\exterior[n]{V}$ inherits from~$V$ the structure of a~{\representation{$\glie$}} via
  \[
    x.(v_1 \wedge \dotsb \wedge v_n)
    =
    \sum_{i=1}^n v_1 \wedge \dotsb \wedge (x.v_i) \wedge \dotsb \wedge v_n
  \]
  for all~$x \in \glie$ and every simple wedge~$v_1 \wedge \dotsb \wedge v_n \in \exterior[n]{V}$.
\end{lemma}

\begin{proof}
  For readability we will denote for all~$v_1, \dotsc, v_n \in V$ the corresponding simple wedge~$v_1 \wedge \dotsb \wedge v_n$ by~$v_1 \dotsm v_n$.
  We first need to show that every~$x \in \glie$ the proposed action on~$x$ on~$\exterior[n]{V}$ is well-defined and linear.
  We do so by using the universal property of the exterior power.
  For this we consider the map
  \[
    \gamma_x
    \colon
    V \times \dotsb \times V
    \to
    \exterior[n]{V} \,,
    \quad
    (v_1, \dotsc, v_n)
    \mapsto
    \sum_{i=1}^n v_1 \dotsm (x.v_i) \dotsm v_n  \,.
  \]
  This map is multilinear and we need to check that it is alternating.
  So let~$v_1, \dotsc, v_n \in V$ with~$v_i = v_j$ for some~$i < j$.
  Then~$v_1 \dotsm (x.v_k) \dotsm v_n = 0$ whenever~$k \neq i,j$ and thus
  \begin{align*}
    \gamma_x(v_1, \dotsc, v_n)
    ={}&
    \sum_{k=1}^n v_1 \dotsm (x.v_k) \dotsm v_n
    \\
    ={}&
      v_1 \dotsm (x.v_i) \dotsm v_j \dotsm v_n
    + v_1 \dotsm v_i \dotsm (x.v_j) \dotsm v_n
    \\
    ={}&
      v_1 \dotsm (x.v_i + v_i) \dotsm (x.v_j + v_j) \dotsm v_n
    \\
    {}&
    - v_1 \dotsm (x.v_i) \dotsm (x.v_j) \dotsm v_n
    \\
    {}&
    - v_1 \dotsm v_i \dotsm v_j \dotsm v_n
    \\
    ={}&
    0 \,.
  \end{align*}
  It now follows from the universal property of the exterior power~$\exterior[n]{V}$ that~$\gamma_x$ induces a well-defined linear map
  \[
    \exterior[n]{V}
    \to
    \exterior[n]{V} \,,
    \quad
    v_1 \dotsm v_n
    \mapsto
    \sum_{i=1}^n v_1 \dotsb (x.v_i) \dotsb v_n  \,.
  \]

  We need to check that this action of~$\glie$ on~$\exterior[n]{V}$ is again compatible with the Lie~bracket of~$\glie$.
  We see that
  \begin{align*}
    {}&
    x.y.(v_1 \dotsm v_n)
    \\
    ={}&
    x. \sum_{i=1}^n v_1 \dotsm (y.v_i) \dotsm v_n
    \\
    ={}&
    \sum_{i < j}  v_1 \dotsm (x.v_i) \dotsm (y.v_j) \dotsm v_n
    \\
    {}&
    + \sum_{i=1}^n  v_1 \dotsm (x.y.v_i) \dotsm v_n
    \\
    {}&
    + \sum_{i < j}  v_1 \dotsm (y.v_i) \dotsm (x.v_j) \dotsm v_n
  \end{align*}
  and hence
  \begin{align*}
    {}&
    x.y.(v_1 \dotsm v_n) - y.x.(v_1 \dotsm v_n)
    \\
    ={}&
    \sum_{i=1}^n  v_1 \dotsm (x.y.v_i) \dotsm v_n
    -
    \sum_{i=1}^n  v_1 \dotsm (y.x.v_i) \dotsm v_n
    \\
    ={}&
    \sum_{i=1}^n  v_1 \dotsm (x.y.v_i - y.x.v_i) \dotsm v_n
    \\
    ={}&
    \sum_{i=1}^n  v_1  \dotsm ([x,y].v_i) \dotsm v_n
  \end{align*}
  for all~$x, y \in \glie$ and every simple wedge~$v_1 \dotsm v_n \in \exterior[n]{V}$.
\end{proof}

It follows from \cref{induced exterior action} that the natural action of~$\sllie(V)$ on~$V$ induces on~$\exterior[2]{V}$ the structure of an~{\representation{$\sllie(V)$}} via
\[
  X.(v_1 \wedge v_2)
  =
  (X.v_1) \wedge v_2 + v_1 \wedge (X.v_2)
\]
for all~$X \in \sllie(V)$ and every simple wedge~$v_1 \wedge v_2 \in \exterior[2]{V}$.
This action of~$\sllie(V)$ on~$\exterior[2]{V}$ corresponds to a homomorphism of Lie~algebras~$\rho \colon \sllie(V) \to \gllie(\exterior[2]{V})$ that is given by~$\rho(X)(x) = X.x$ for all~$X \in \sllie(V)$ and~$x \in \exterior[2]{V}$.
We show in the following that~$\rho$ restricts to an isomorphism~$\sllie(V) \to \glie_\beta$.

We first observe that the image of~$\rho$ is contained in~$\glie_\beta$:
We need to show that
\[
  \beta(\rho(X)(x), y) + \beta(x, \rho(X)(y)
  =
  0
\]
for all~$X \in \sllie(V)$ and~$x, y \in \exterior[2]{V}$.
It sufficies to consider the case that both~$x$ and~$y$ are simple wedges~$x = v_1 \wedge v_2$ and~$y = v_3 \wedge v_4$.
Then the term
\begin{align*}
  {}&
  \gamma(v_1, v_2, v_3, v_4)
  \\
  \defined{}&
    \beta(\rho(X)(v_1 \wedge v_2), v_3 \wedge v_4)
  + \beta(v_1 \wedge v_2, \rho(X)(v_3 \wedge v_4))
  \\
  ={}&
    \rho(X)(v_1 \wedge v_2) \wedge v_3 \wedge v_4
  + v_1 \wedge v_2 \wedge \rho(X)(v_3 \wedge v_4)
  \\
  ={}&
    (X.v_1) \wedge v_2 \wedge v_3 \wedge v_4
  + v_1 \wedge (X.v_2) \wedge v_3 \wedge v_4
  \\
  {}&
  + v_1 \wedge v_2 \wedge (X.v_3) \wedge v_4
  + v_1 \wedge v_2 \wedge v_3 \wedge (X.v_4)
\end{align*}
is multilinear and alternating in~$v_1$,~$v_2$,~$v_3$,~$v_4$.
To show that~$\gamma(v_1, v_2, v_3, v_4) = 0$ for all~$v_1, v_2, v_3, v_4 \in V$ it therefore sufficies to show that~$\gamma(e_1, e_2, e_3, e_4) = 0$ for the given basis~$e_1$,~$e_2$,~$e_3$,~$e_4$ of~$V$.
Let~$A \in \sllie_4(\Complex)$ be the matrix that represents~$X \in \sllie(V)$ with respect to this basis, which means that~$X.e_j = \sum_{i=1}^4 A_{ij} e_i$ for every~$j = 1, 2, 3, 4$.
Then
\[
  (X.e_1) \wedge e_2 \wedge e_3 \wedge e_4
  =
  \sum_{i=1}^4 A_{i1} e_i \wedge e_2 \wedge e_3 \wedge e_4
  =
  A_{11} e_1 \wedge e_2 \wedge e_3 \wedge e_4
\]
and similarly
\begin{align*}
  e_1 \wedge (X.e_2) \wedge e_3 \wedge e_4
  &=
  A_{22} e_1 \wedge e_2 \wedge e_3 \wedge e_4 \,,
  \\
  e_1 \wedge e_2 \wedge (X.e_3) \wedge e_4
  &=
  A_{33} e_1 \wedge e_2 \wedge e_3 \wedge e_4 \,,
  \\
  e_1 \wedge e_2 \wedge e_3 \wedge (X.e_4)
  &=
  A_{44} e_1 \wedge e_2 \wedge e_3 \wedge e_4 \,.
\end{align*}
Hence altogether
\begin{align*}
  \gamma(e_1, e_2, e_3, e_4)
  &=
  (A_{11} + A_{22} + A_{33} + A_{44}) e_1 \wedge e_2 \wedge e_3 \wedge e_4
  \\
  &=
  \tr(A) e_1 \wedge e_2 \wedge e_3 \wedge e_4
  =
  0 \,.
\end{align*}
We have shown that the image of~$\rho$ lies in~$\glie_\beta$.

It remains to show that the homomorphism~$\rho$ is both surjective onto~$\glie_\beta$ and injective.
We know that~$\dim \sllie(V) = \dim \sllie_4(\Complex) = 4^2 - 1 = 15$ and~$\dim \glie_\beta = \dim \solie_6(\Complex) = 15$.
It therefore sufficies to show that~$\rho$ is injective.
It moreover sufficies to show that~$\rho$ is nonzero because~$\sllie(V)$ is simple.
If we consider the endomorphism~$X \in \sllie(V)$ with~$X(e_1) = e_1$,~$X(e_2) = -e_2$ and~$X(e_3) = X(e_4) = 0$ then
\[
  \rho(X)(e_1 \wedge e_3)
  =
  (X.e_1) \wedge e_3 + e_1 \wedge (X.e_3)
  =
  e_1 \wedge e_3 + e_1 \wedge 0
  =
  e_1 \wedge e_3
  \neq
  0
\]
whence~$\rho(X) \neq 0$.




