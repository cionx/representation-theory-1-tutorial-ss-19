\section{}





\addtocounter{subsection}{1}





\addtocounter{subsection}{1}





\subsection{}

We follow the hint:
We have for every~$v \in V$ and every simple wedge~$w_1 \wedge w_2 \in \exterior[2]{V}$ that
\[
  v \wedge w_1 \wedge w_2
  =
  - w_1 \wedge v \wedge w_2
  =
  - w_1 \wedge w_2 \wedge v \,.
\]
The bilinear map
\[
  \beta
  \colon
  \exterior[2]{V}
  \times
  \exterior[2]{V}
  \to
  \exterior[4]{V} \,,
  \quad
  (x,y)
  \mapsto
  x \wedge y
\]
is hence symmetric, beause
\[
  \beta(v_1 \wedge v_2, w_1 \wedge w_2)
  =
  v_1 \wedge v_2 \wedge w_1 \wedge w_2
  =
  v_1 \wedge w_1 \wedge w_2 \wedge v_2
  =
  w_1 \wedge w_2 \wedge v_1 \wedge v_1
\]
for all simple wedges~$v_1 \wedge v_2, w_1 \wedge w_2 \in \exterior[2]{V}$.
The bilinear form~$\beta$ is also non-degenerate:
Let~$e_1$,~$e_2$,~$e_3$,~$e_4$ be a basis of~$V$.
Then a basis of~$\exterior[2]{V}$ is given by the simple wedges
\[
  e_1 \wedge e_2  \,,
  \quad
  e_1 \wedge e_3  \,,
  \quad
  e_1 \wedge e_4  \,,
  \quad
  e_2 \wedge e_3  \,,
  \quad
  e_2 \wedge e_4  \,,
  \quad
  e_3 \wedge e_4  \,.
\]
We identify the exterior power~$\exterior[4]{V}$ with the ground field~$k$ via the single basis element~$e_1 \wedge e_2 \wedge e_3 \wedge e_4$.
With respect to this basis of~$\exterior[2]{V}$ the bilinear form~$\beta$ is then given by the following matrix:
\[
  \begin{pmatrix*}[r]
      &     &   &   &     & 1 \\
      &     &   &   & -1  &   \\
      &     &   & 1 &     &   \\
      &     & 1 &   &     &   \\
      & -1  &   &   &     &   \\
    1 &     &   &   &     &  
  \end{pmatrix*}
\]
This matrics is invertible, which means that~$\beta$ is non-degenerate.

\begin{warning}
  The exterior power~$\exterior[4]{V}$ is \emph{not} the same---and not even isomorphic to---the iterated exterior power~$\exterior[2]{\exterior[2]{V}}$.
\end{warning}

We know from the first exercise sheet that
\[
  \glie_\beta
  \defined
  \biggl\{
    \varphi
    \in
    \gllie(V)
    \suchthat[\bigg]
    \text{$\beta(\varphi(x), y) + \beta(x, \varphi(y)) = 0$ for all~$x,y \in \exterior[2]{V}$}
  \biggr\}
\]
is a Lie~subalgebra of~$\gllie(V)$ that is isomorphic to~$\solie_6(\Complex)$, because~$\beta$ in non-degenerate and symmetric and~$\exterior[2]{V}$ is~\dash{six}{dimensional}.
We will therefore construct an isomorphism~$\sllie_4(V) \cong \glie_\beta$, where~$\sllie(V) \cong \sllie_4(\Complex)$ because~$V$ is~\dash{four}{dimensional}.

\begin{lemma}
  \label{induced exterior action}
  If~$V$ is a representation of a Lie~algebra~$\glie$ then any exterior power~$\exterior[n]{V}$ inherits from~$V$ the structure of a~{\representation{$\glie$}} via
  \[
    x.(v_1 \wedge \dotsb \wedge v_n)
    =
    \sum_{i=1}^n v_1 \wedge \dotsb \wedge (x.v_i) \wedge \dotsb v_n
  \]
  for all~$x \in \glie$ and every simple wedge~$v_1 \wedge \dotsb \wedge v_n \in \exterior[n]{V}$.
\end{lemma}

\begin{proof}
  It follows from the functoriality of the exterior power that for every fixed element~$x \in \glie$ the proposed action of~$x$ on~$\exterior[n]{V}$ is well-defined.
  We need to check that this action is compatible with the Lie~bracket of~$\glie$.
  We see that
  \begin{align*}
    {}&
    x.y.(v_1 \wedge \dotsb \wedge v_n)
    \\
    ={}&
    x. \sum_{i=1}^n v_1 \wedge \dotsb \wedge (y.v_i) \wedge \dotsb v_n
    \\
    ={}&
    \sum_{i < j}  v_1 \wedge \dotsb \wedge (x.v_i) \dotsb \wedge (y.v_j) \wedge \dotsb v_n
    \\
    {}&
    + \sum_{i=1}^n  v_1 \wedge \dotsb \wedge (y.x.v_i) \dotsb \wedge v_j \wedge \dotsb v_n
    \\
    {}&
    + \sum_{i > j}  v_1 \wedge \dotsb \wedge (y.v_j) \dotsb \wedge (x.v_i) \wedge \dotsb v_n
  \end{align*}
  and hence
  \begin{align*}
    {}&
    x.y.(v_1 \wedge \dotsb \wedge v_n) - y.x.(v_1 \wedge \dotsb \wedge v_n)
    \\
    ={}&
    \sum_{i=1}^n  (v_1 \wedge \dotsb \wedge (y.x.v_i) \dotsb \wedge v_j \wedge \dotsb v_n)
    -
    \sum_{i=1}^n  v_1 \wedge \dotsb \wedge (x.y.v_i) \dotsb \wedge v_j \wedge \dotsb v_n
    \\
    ={}&
    \sum_{i=1}^n  v_1 \wedge \dotsb \wedge (y.x.v_i - x.y.v_i) \dotsb \wedge v_j \wedge \dotsb v_n
    \\
    ={}&
    \sum_{i=1}^n  v_1  \wedge \dotsb \wedge ([x,y].v_i) \dotsb \wedge v_j \wedge \dotsb v_n
  \end{align*}
  for all~$x, y \in \glie$ and every simple wedge~$v_1 \wedge \dotsb \wedge v_n \in \exterior[n]{V}$.
\end{proof}

It follows from \cref{induced exterior action} that the natural action of~$\sllie(V)$ on~$V$ induces on~$\exterior[2]{V}$ the structure of a~{\representation{$\sllie(V)$}} via
\[
  X.(v_1 \wedge v_2)
  =
  (X.v_1) \wedge v_2 + v_1 \wedge (x.V_2)
\]
for every~$X \in \sllie(V)$ and every simple wedge~$v_1 \wedge v_2 \in \exterior[2]{V}$.
This action of~$\sllie(V)$ on~$\exterior[2]{V}$ corresponds to a homomorphis of Lie~algebras~$\rho \colon \sllie(V) \to \exterior[2]{V}$ given by~$\rho(X)(x) = X.x$ for all~$X \in \sllie(V)$ and~$x \in \exterior[2]{V}$.
We show in the following that~$\rho$ restricts to an isomorphism~$\sllie(V) \to \glie_\beta$.

We first observe that the image of~$\rho$ is contained in~$\glie_\beta$:
We need to show that
\[
  \beta(\rho(X)(x), y) + \beta(X, \rho(X)(y)
  =
  0
\]
for all~$x, y \in \exterior[2]{V}$.
It sufficies to consider the cases that~$x$ and~$y$ are simple wedges~$x = v_1 \wedge v_2$ and~$y = v_3 \wedge v_4$.
Then
\begin{align*}
  {}&
  \gamma(v_1, v_2, v_3, v_4)
  \\
  ={}&
    \beta(\rho(X)(v_1 \wedge v_2), v_3 \wedge v_4)
  + \beta(v_1 \wedge v_2, \rho(X)(v_3 \wedge v_4))
  \\
  ={}&
    \beta((X.v_1) \wedge v_2 + v_1 \wedge (X.v_2), v_3 \wedge v_4)
  + \beta(v_1 \wedge v_2, (X.v_3) \wedge v_n + v_3 \wedge (X.v_4))
  \\
  ={}&
    (X.v_1) \wedge v_2 \wedge v_3 \wedge v_4
  + v_1 \wedge (X.v_2) \wedge v_3 \wedge v_4
  \\
  {}&
  + v_1 \wedge v_2 \wedge (X.v_3) \wedge v_4
  + v_1 \wedge v_2 \wedge v_3 \wedge (X.v_4)
\end{align*}
is multilinear and alternating in~$v_1$,~$v_2$,~$v_3$,~$v_4$.
To show that~$\gamma(v_1, v_2, v_3, v_4) = 0$ for all~$v_1, v_2, v_3, v_4 \in V$ it therefore sufficies to show that~$\gamma(e_1, e_2, e_3, e_4) = 0$ for a basis~$e_1, e_2, e_3, e_n$ of~$V$.
Let~$A \in \sllie_4(\Complex)$ be the matrix that represents~$X \in \sllie(V)$ with respect to this basis, which means that~$X.e_j = \sum_{i=1}^n A_{ij} e_i$ for every~$j = 1, \dotsc, n$.
Then
\[
  (X.e_1) \wedge e_2 \wedge e_3 \wedge e_4
  =
  \sum_{i=1}^n A_{i1} e_i \wedge e_2 \wedge e_3 \wedge e_4
  =
  A_{11} e_1 \wedge e_2 \wedge e_3 \wedge e_4
\]
and similarly
\begin{align*}
  e_1 \wedge (X.e_2) \wedge e_3 \wedge e_4
  &=
  A_{22} e_1 \wedge e_2 \wedge e_3 \wedge e_4 \,,
  \\
  v_1 \wedge v_2 \wedge (X.v_3) \wedge v_4
  &=
  A_{33} e_1 \wedge e_2 \wedge e_3 \wedge e_4 \,,
  \\
  v_1 \wedge v_2 \wedge v_3 \wedge (X.v_4)
  &=
  A_{44} e_1 \wedge e_2 \wedge e_3 \wedge e_4 \,.
\end{align*}
Hence altogether
\begin{align*}
  \gamma(e_1, e_2, e_3, e_4)
  &=
  (A_{11} + A_{22} + A_{33} + A_{44}) e_1 \wedge e_2 \wedge e_3 \wedge e_4
  \\
  &=
  \tr(A) e_1 \wedge e_2 \wedge e_3 \wedge e_4
  =
  0 \,.
\end{align*}
We have shown that the image of~$\rho$ lies in~$\glie_\beta$.

We know that~$\dim \sllie(V) = \dim \sllie_4(\Complex) = 4^2 - 1 = 15$ and~$\dim \glie_\beta = \dim \solie_6(\Complex) = 15$.
It therefore now sufficies to show that~$\rho$ is injective.
Since~$\sllie(V)$ is simple if sufficies to show that~$\rho$ is nonzero.
For this we consider the endomorphism~$X \in \sllie(V)$ with~$X(e_1) = e_1$,~$X(e_2) = -e_2$ and~$X(e_3) = X_(e_4) = 0$ for some basis~$e_1$,~$e_2$,~$e_3$,~$e_4$ of~$V$.
Then
\[
  \rho(X)(e_1 \wedge e_3)
  =
  (X.e_1) \wedge e_3 + e_1 \wedge (X.e_3)
  =
  e_1 \wedge e_3 + e_1 \wedge 0
  =
  e_1 \wedge e_3
  \neq
  0
\]
and hence~$\rho(X) \neq 0$.
This shows that~$\rho$ is nonzero.










