\section{}





\subsection{}

To construct a Lie algebra homomorphism~$\varphi \colon \glie \to \sllie_{n+1}(\Complex)$ we only need to specify the images~$E_i$,~$H_i$,~$F_i$ of the generators~$e_i$,~$h_i$,~$f_i$ and then check that these proposed images satisfy the given relations~(22)--(27).
(This is what it means for the Lie~algebra~$\glie$ to be given by the generators~$\{ e_i, h_i, f_i \}_{i=1}^n$ and relations~(22)--(27).)

In the case~$n = 1$ we know that the Lie~algebra~$\sllie_2(\Complex)$ has the standard basis~$e$,~$f$,~$h$ constisting of the matrices
\[
  e
  =
  \begin{pmatrix}
    0 & 1 \\
    0 & 0
  \end{pmatrix} \,,
  \quad
  f
  =
  \begin{pmatrix}
    0 & 0 \\
    1 & 0
  \end{pmatrix} \,,
  \quad
  h
  =
  \begin{pmatrix*}[r]
    1 &  0  \\
    0 & -1
  \end{pmatrix*}  \,.
\]
These three matrices satisfy the relations~$[e,f] = h$,~$[h,e] = 2e$ and~$[h,f] = -2f$, which corresponds to the relations~(23),~(24),~(25).
Motivated by this example we choose for any~$n \geq 1$ the images~$E_i, F_i, H_i \in \sllie_{n+1}(\Complex)$ as
\[
  E_i
  \defined
  E_{i,i+1} \,,
  \quad
  F_i
  \defined
  E_{i+1,i} \,,
  \quad
  H_i
  \defined
  E_{ii} - E_{i+1,i+1}  \,.
\]
We need to check that these elements satisfy the relations~(22)--(27):
\begin{enumerate}[label=(\arabic*),start=22]
  \item
    We find~$[H_i, H_j] = 0$ because both~$H_i$ and~$H_j$ are diagonal matrices and hence commute with each other.
  \item
    We find
    \begin{align*}
      [E_i, F_j]
      &=
      [E_{i,i+1}, E_{j+1,j}]
      \\
      &=
      E_{i,i+1} E_{j+1,j} - E_{j+1,j} E_{i,i+1}
      \\
      &=
      \delta_{i+1,j+1} E_{ij} - \delta_{ij} E_{j+1,i+1}
      \\
      &=
      \delta_{ij} E_{ij} - \delta_{ij} E_{j+1,i+1}
      \\
      &=
      \delta_{ij} E_{ii} - \delta_{ij} E_{i+1,i+1}
      \\
      &=
      \delta_{ij} (E_{ii} - E_{i+1,i+1})
      \\
      &=
      \delta_{ij} H_i \,.
    \end{align*}
  \item
    \label{H and E}
    We find
    \begin{align*}
      [H_i, E_j]
      &=
      [E_{ii} - E_{i+1,i+1}, E_{j,j+1}]
      \\
      &=
      (E_{ii} - E_{i+1,i+1}) E_{j,j+1} - E_{j,j+1} (E_{ii} - E_{i+1,i+1})
      \\
      &=
      E_{ii} E_{j,j+1} - E_{i+1,i+1} E_{j,j+1} - E_{j,j+1} E_{ii} + E_{j,j+1} E_{i+1, i+1}
      \\
      &=
      \delta_{ij} E_{i,j+1} - \delta_{i+1,j} E_{i+1,j+1} - \delta_{i,j+1} E_{ji} + \delta_{i+1,j+1} E_{j,i+1}
      \\
      &=
      \delta_{ij} E_{i,j+1} - \delta_{i+1,j} E_{i+1,j+1} - \delta_{i,j+1} E_{ji} + \delta_{ij} E_{j,i+1}
      \\
      &=
      \delta_{ij} (E_{i,j+1} + E_{j,i+1}) - \delta_{i+1,j} E_{i+1,j+1} - \delta_{i,j+1} E_{ji}
      \\
      &=
      2 \delta_{ij} E_{j,j+1} - \delta_{i+1,j} E_{j,j+1} - \delta_{i,j+1} E_{j,j+1}
      \\
      &=
      (2 \delta_{ij} - \delta_{i+1,j} - \delta_{i,j+1}) E_{j,j+1}
      \\
      &=
      a_{ji} E_j  \,.
    \end{align*}
  \item
    \label{H and F}
    This relation can be checked in the same way as~(24).
    But we can also observe that~$F_j = E_j^T$ and~$H_i = H_i^T$ and hence
    \[
      [H_i, F_j]
      =
      [H_i^T, E_j^T]
      =
      [E_j, H_i]^T
      =
      a_{ji} E_j^T
      =
      -a_{ji} F_j \,.
    \]
  \item
    We find
    \begin{align*}
      \ad(E_i)(E_j)
      &=
      [E_i, E_j]
      \\
      &=
      [E_{i,i+1}, E_{j,j+1}]
      \\
      &=
      E_{i,i+1} E_{j,j+1} - E_{j,j+1} E_{i,i+1}
      \\
      &=
      \delta_{i+1,j} E_{i,j+1} - \delta_{i,j+1} E_{j,i+1} \,.
    \end{align*}
    If~$\abs{i-j} \geq 2$ then~$\delta_{i+1,j} = \delta_{i,j+1} = 0$ and~$a_{ji} = 0$ and hence
    \[
      \ad(E_i)^{-a_{ji}+1}(E_j)
      =
      \ad(E_i)(E_j)
      =
      0
    \]
    as needed.
    If~$i = j-1$ then~$[E_i, E_j] = \ad(E_i)(E_j) = E_{i,j+1} = E_{i,i+2}$ and~$a_{ji} = -1$ and thus
    \begin{align*}
      \ad(E_i)^{-a_{ji}+1}(E_j)
      &=
      \ad(E_i)^2(E_j)
      \\
      &=
      [E_i, [E_i, E_j]]
      \\
      &=
      [E_{i,i+1}, E_{i,i+2}]
      \\
      &=
      E_{i, i+1} E_{i, i+2} - E_{i,i+2} E_{i,i+1}
      \\
      &=
      \delta_{i,i+1} E_{i,i+2} - \delta_{i,i+2} E_{i,i+1}
      \\
      &=
      0 \,.
    \end{align*}
    The case~$i = j+1$ works the same.
  \item
    This can be done by similar calculations as in~(26) but can also be directly derived from~(26) by again using the matrix transpose.
\end{enumerate}


\begin{remark}
  \leavevmode
  \begin{enumerate}
    \item
      We used in~\ref*{H and F} that~that~$[A,B]^T = (AB - BA)^T = B^T A^T - A^T B^T = [B^T, A^T]$.
    \item
      For~\ref*{H and E} and~\ref*{H and F} it is useful to understand how a commutator~$[D,A]$ looks like if~$D$ is a diagonal matrix with diagonal entries~$\lambda_1, \dotsc, \lambda_n \in \Complex$:
      
      The matrix~$DA$ results from~$A$ by multiplying for every~$i = 1, \dotsc, n$ the~\dash{$i$}{th} row of~$A$ by the corresponding diagonal entry~$\lambda_i$.
      Similarly the matrix~$AD$ results from~$A$ by multiplying for every~$j = 1, \dotsc, n$ the~\dash{$j$}{th} column of~$A$ by the corresponding diagonal entry~$\lambda_j$.
      This means in formulae that
      \[
        (DA)_{ij}
        =
        \lambda_i A_{ij}
        \quad\text{and}\quad
        (AD)_{ij}
        =
        \lambda_j A_{ij}
      \]
      for all~$i,j = 1, \dotsc, n$, and thus
      \[
        [D,A]_{ij}
        =
        (DA - AD)_{ij}
        =
        (DA)_{ij} - (AD)_{ij}
        =
        \lambda_i A_{ij} - \lambda_j A_{ij}
        =
        (\lambda_i - \lambda_j) A_{ij}
      \]
      for all~$i, j = 1, \dotsc, n$.
      
      It follows that
      \begin{gather*}
        [H_i, E_j]
        =
        [H_i, E_{j,j+1}]
        =
        ((H_i)_{jj} - (H_i)_{j+1,j+1}) E_{j,j+1}
      \shortintertext{where}
        \begin{aligned}
          (H_i)_{jj} - (H_i)_{j+1,j+1}
          &=
          \begin{cases}
            \phantom{-}0  & \text{if~$j+1 < i$} \,, \\
                      -1  & \text{if~$j+1 = i$} \,, \\
            \phantom{-}2  & \text{if~$j = i$} \,,   \\
                      -1  & \text{if~$j = i+1$} \,, \\
            \phantom{-}0  & \text{if~$j > i+1$} \,.
          \end{cases}
          \\
          &= a_{ji}
        \end{aligned}
      \end{gather*}
      Hence~$[H_i, E_j] = a_{j,i} E_j$ as desired.
  \end{enumerate}
\end{remark}

We have shown that the matrices~$E_i, F_i, H_i$ satisfy the given relations.
There hence exists a unique homomorphism of Lie~algebras~$\varphi \colon \glie \to \sllie_{n+1}(\Complex)$ with~$\varphi(e_i) = E_i$,~$\varphi(f_i) = F_i$ and~$\varphi(h_i) = H_i$.
It remains to show that~$\varphi$ is surjective.

The image of~$\varphi$ is a Lie~subalgebra of~$\sllie_{n+1}(\Complex)$.
It therefore sufficies to show that~$\sllie_{n+1}(\Complex)$ is generated by the elements~$\{E_i, F_i, H_i\}_{i=1}^n$ as a Lie~algebra.
Let~$\slie$ be the Lie~subalgebra of~$\sllie_{n+1}(\Complex)$ generated by these elements.
We know that~$\sllie_{n+1}(\Complex)$ has as a basis the diagonal matrices~$H_1, \dotsc, H_n$ together with the off-diagonal matrices~$E_{ij}$ where~$1 \leq i \neq j \leq n+1$.
It sufficies to show that these matrices are contained in~$\slie$.
This holds for~$H_1, \dotsc, H_n$ by construction of~$\slie$.

Let us consider the off-diagonal matrices~$E_{ij}$ with~$j > i$.
We fix the index~$i$ and show that~$E_{i,i+1}, E_{i,i+2}, \dotsc, E_{i,n+1} \in \slie$.
This holds for~$E_{i,i+1} = E_i$ by construction of~$\slie$.
If~$E_{ij} \in \slie$ for some~$i+1 \leq j < n+1$ then we find inductively that the matrix
\[
  [E_{ij}, E_j]
  =
  [E_{ij}, E_{j,j+1}]
  =
  E_{ij} E_{j,j+1} - E_{j,j+1} E_{ij}
  =
  E_{i,j+1}
\]
is again contained~$\slie$.
This shows that all off-diagonal matrices~$E_{ij}$ with~$j > i$ are contained in~$\slie$.

For the off-diagonal matrices~$E_{ij}$ with~$i < j$ we can argue in the same way by using the matrices~$F_i$ instead of~$E_i$.
But we could also observe that the Lie~algebra generating set~$\{E_i, F_i, H_i\}_{i=1}^n$ of~$\slie$ is closed under matrix transposition whence~$\slie$ is closed under matrix transposition (because matrix transposition is a Lie algebra anti-automorphism).
It thus follows for all~$i < j$ from~$E_{ji} \in \slie$ that also~$E_{ij} = E_{ji}^T \in \slie$.





\subsection{}

We construct an inverse~$\psi \colon \sllie_2(\Complex) \to \glie$ to~$\varphi$.
We define~$\psi$ to be the unique linear map with~$\psi(e) = e_1$,~$\psi(f) = f_1$ and~$\psi(h) = h_1$.
Recall that~$[h,e] = 2e$,~$[h,f] = -2f$ and~$[e,f] = h$.
The relations~(23),~(24),~(25) therefore ensure that~$\psi$ is a homomorphism of Lie~algebras.
Then~$\psi \varphi = \id_{\sllie_2(\Complex)}$ because this holds on the basis~$e$,~$h$,~$f$ of~$\sllie_2(\Complex)$ and~$\varphi \psi = \id_{\glie}$ because this holds on the Lie~algebra generators~$e_1$,~$h_1$,~$f_1$ of~$\glie$.



