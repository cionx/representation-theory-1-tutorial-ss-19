\section{}

All occuring field are of characteristic zero.





\addtocounter{subsection}{1}





\addtocounter{subsection}{1}





\subsection{}

To show that~$I \defined [\glie, \glie]^\perp$ is contained in~$\rad(\glie)$ we show that~$I$ is a solvable ideal in~$\glie$.
The claimed inclusion then follows because~$\rad(\glie)$ is the unique maximal solvable ideal in~$\glie$.

To see that~$I$ is an ideal in~$\glie$ we observe that~$[\glie, \glie]$ is an ideal in~$\glie$, and that for every ideal~$J$ in~$\glie$ the orthogonal~$J^\perp$ is again an ideal in~$\glie$ because the Killing form~$\kappa_{\glie}$ is associative.%
\footnote{This was hopefully shown in the lecture for the proof of the equivalence of the characterizations for semisimple Lie~algebras.}

To see that~$I$ is solvable we observe that~$\kappa_{\glie}(x,y) = 0$ for all~$x \in I$ and~$y \in [\glie, \glie]$ by definition of~$I$ and hence especially~$\kappa_{\glie}(x,y) = 0$ for all~$x \in I$ and~$y \in [I,I]$.
We know that the restriction of~$\kappa_{\glie}$ to~$I$ coincides with the Killing form~$\kappa_I$ because~$I$ is an ideal in~$\glie$.
We have thus found that~$\kappa_I(x,y) = 0$ for all~$x \in I$ and~$y \in [I,I]$.
It follows from Cartan’s~criterion (part~2 of this exercise) that the ideal~$I$ is indeed solvable.

To show that~$\rad(\glie)$ is contained in~$[\glie, \glie]^\perp$ we need to show that~$\kappa_{\glie}(\rad(\glie), [\glie, \glie]) = 0$.
It follows from the associativity of~$\kappa_{\glie}$ that
\[
  \kappa_{\glie}(\rad(\glie), [\glie, \glie])
  =
  \kappa_{\glie}([\rad(\glie), \glie], \glie)  \,.
\]
It follows from the upcoming part~4 (that we will prove independent of this part) that the ideal~$[\rad(\glie), \glie]$ is nilpotent.
That~$\kappa_{\glie}([\rad(\glie), \glie], \glie) = 0$ hence follows from the following observation:

\begin{lemma}
  \label{nilpotent is in killing radical}
  Let~$\glie$ be a finite dimensional Lie~algebra and let~$I$ be a nilpotent ideal in~$\glie$.
  Then~$\kappa_{\glie}(I, \glie) = 0$, i.e.~$I$ is contained in the radical of~$\kappa$.
\end{lemma}

\begin{proof}
  Let~$x \in I$ and~$y \in \glie$.
  We have for all~$n \geq 0$ that~$\ad(x)(I^n) \subseteq I^{n+1}$ and~$\ad(y)(I^n) \subseteq I^n$ because~$I^n$ is an ideal in~$I$.
  (We are using the convention~$I^0 = I$.)
  It hence follows by induction that the image of~$(\ad(x)\ad(y))^{n+1}$ is contained in~$I^n$ for every~$n \geq 0$.

  It therefore follows for~$n$ sufficiently large from~$I^n = 0$ that also~$(\ad(x)\ad(y))^n = 0$ whence~$\ad(x)\ad(y)$ is nilpotent.
  Therefore~$\kappa_{\glie}(x,y) = \tr(\ad(x)\ad(y)) = 0$.
\end{proof}



\subsection{}
\label{complicated part}

This solution is longer than necessary but hopefully sheds some light on what’s going on.
We are taking the technical aspects of our approach from \cite[I.{\S}5.3]{bourbaki_lie} and \cite[\S19.5]{tauvel_yu}.

\begin{recall}
  As a motivation for the upcoming definition of~$\srad(\glie)$ we recall the equivalent characterizations of the Jacobson radical~$\jacob(A)$ of a finite dimensional~{\algebra{$k$}}~$A$:
  \begin{enumerate}
    \item
      The intersection of all maximal left ideal of~$A$.
%     \item
%       The intersection of all maximal right ideals of~$A$.
    \item
      The unique maximal nilpotent ideal of~$A$.
    \item
      The set of all~$x \in A$ that annihilate every simple left~{\module{$A$}}.
%     \item
%       The set of all~$x \in A$ that annihilate every simple right~{\module{$A$}}.
  \end{enumerate}
  Moreover:
  \begin{enumerate}[resume]
    \item
      The algebra~$A$ is semisimple if and only if~$\jacob(A) = 0$.
  \end{enumerate}
\end{recall}

For a finite dimensional Lie~algebra~$\glie$ one can similarly consider four kinds of \enquote{radical}:
\begin{enumerate}
  \item
    The sum of any two solvable ideals of~$\glie$ is again solvable whence~$\glie$ admits a unique maximal solvable ideal.
    This is the \emph{radical}~$\rad(\glie)$ as introduced in the lecture.
    The Lie~algebra~$\glie$ is semisimple if and only if~$\rad(\glie) = 0$.
  \item
    The \emph{radical of the Killing~form} of~$\glie$.
    Note that by Cartan’s~criterion,~$\rad(\kappa)$ is solvable.
    (The restriction of~$\kappa_{\glie}$ to~$\rad(\kappa_{\glie})$ is the Killing~form of~$\rad(\kappa_{\glie})$ because~$\rad(\kappa_{\glie})$ is an ideal in~$\glie$.
    But this restriction is zero by definition of~$\rad(\kappa_{\glie})$ whence~$\kappa_{\rad(\kappa_{\glie})} = 0$.
    Now apply Cartan’s criterion to~$\rad(\kappa_{\glie})$.)
    Thus~$\rad(\kappa_{\glie})$ is contained in~$\rad(\glie)$.
  \item
    The sum of any two nilpotent ideals of~$\glie$ is again nilpotent whence~$\glie$ admits a unique maximal nilpotent ideal.
    (This was not shown in the lecture.
    We will also not show this here, as we won’t need this.)
    This is the \emph{nilradical}~$\nil(\glie)$.
    Note that~$\nil(\glie) \subseteq \rad(\glie)$ because every nilpotent ideal is in particular a solvable ideal.
    Even better, by \cref{nilpotent is in killing radical} we have~$\nil(\glie) \subseteq \rad(\kappa_{\glie})$.
  \item
    The set
    \[
      \srad(\glie)
      \defined
      \left\{
        x \in \glie
      \suchthat*
        \begin{tabular}{@{}c@{}}
          $x$ annihilates every finite dimensional \\
          irreducible~{\representation{$\glie$}}
        \end{tabular}
      \right\}
    \]
    is called the \emph{nilpotent radical} of~$\glie$.
    The author doesn’t like this name because it sounds too much like the nilradical, and hence will not use it.%
    \footnote{The term is taken from~\cite[I.{\S}5, Definition 3]{bourbaki_lie}, the notation~$\srad(\glie)$ is taken from~$\slie$ used there.}
    Note that~$\srad(\glie)$ is an ideal because
    \[
      \srad(\glie)
      =
      \bigcap
      \left\{
        \ker(\rho)
      \suchthat*
        \begin{tabular}{@{}c@{}}
          $(V,\rho)$ is a finite dimensional \\
          irreducible~{\representation{$\glie$}}
        \end{tabular}
      \right\}  \,.
    \]

\end{enumerate}

We show in the following that~$\srad(\glie)$ is nilpotent (and hence contained in~$\nil(\glie)$) and that~$\srad = \rad(\glie) \cap [\glie, \glie]$.
Then altogether
\[
  \rad(\glie)
  \supseteq
  \rad(\kappa_{\glie})
  \supseteq
  \nil(\glie)
  \supseteq
  \srad
  =
  \rad(\glie) \cap [\glie, \glie] \,.
\]
(We refer to \cite{radical_history} for a short comment regarding the history of these radicals.)
It then follows from~$[\rad(\glie), \glie] \subseteq \rad(\glie) \cap [\glie, \glie] = \srad(\glie)$ that~$[\rad(\glie), \glie]$ is nilpotent, as needed for the exercise.






\subsubsection*{Showing that~$\srad(\glie)$ is nilpotent}

The nilpotency of~$\srad(\glie)$ follows from an equivalent characterization of~$\srad(\glie)$:

\begin{proposition}
  \label{characterization of nilpotent radical}
  For an ideal~$I$ in a Lie~algebra~$\glie$ the following conditions are equivalent:
  \begin{enumerate}
    \item
      \label{annihilates}
      $I$ annihilates every finite dimensional irreducible~{\representation{$\glie$}}~$V$.
    \item
      \label{pointwise nilpotent}
      $I$ acts nilpotent on every finite dimensional~{\representation{$\glie$}}~$(V,\rho)$ in the sense that~$\rho(x)$ is nilpotent for every~$x \in I$.
    \item
      \label{nilpotent}
      $I$ acts nilpotent on every finite dimensional~{\representation{$\glie$}}~$(V,\rho)$ in the sense that for some~$n \geq 1$,~$\rho(x)^n = 0$ for every~$x \in I$.
  \end{enumerate}
  Hence~$\srad(\glie)$ is the unique maximal ideal in~$\glie$ that acts nilpotent on every finite dimensional~{\representation{$\glie$}}.
\end{proposition}

\begin{lemma}
  \label{nilpotent on irrep is zero}
  Let~$V$ be a finite dimensional vector space and let~$\glie$ be a Lie~subalgebra of~$\gllie(V)$ such that~$V$ is irreducible as a~{\representation{$\glie$}}.
  If~$I$ is an ideal in~$\glie$ that consists of nilpotent endomorphisms then~$I = 0$.
\end{lemma}

\begin{proof}
  It follows from Engel’s theorem that~$I$ annihilates some nonzero linear subspace~$U$ of~$V$.
  The subspace~$U$ is a~{\subrepresentation{$\glie$}}:
  For all~$x \in I$,~$y \in \glie$ and~$u \in U$,
  \[
    x.y.u
    =
    y.x.u + [x,y].u
    =
    0
  \]
  because~$x, [x,y] \in I$.
  It follows that~$U = V$ because~$V$ is irreducible.
\end{proof}

\begin{proof}[Proof of \cref{characterization of nilpotent radical}]
  \leavevmode
  \begin{description}
    \item[\ref*{nilpotent}~$\implies$~\ref*{pointwise nilpotent}]
      Okay.
    \item[\ref*{pointwise nilpotent}~$\implies$~\ref*{annihilates}]
      We may apply \cref{nilpotent on irrep is zero} to the ideal~$\rho(I)$ of~$\rho(\glie)$ to find~$\rho(I) = 0$.
    \item[\ref*{annihilates}~$\implies$~\ref*{nilpotent}]
      There exists by the finite dimensionality of~$V$ a filtration
      \[
        0
        =
        V_0
        \subsetneq
        V_1
        \subsetneq
        \dotsb
        \subsetneq
        V_n
        =
        V
      \]
      by~{\subrepresentations{$\glie$}} of maximal length, i.e.\ such that the quotients~$V_i/V_{i-1}$ are irreducible.
      Then~$I.(V_i/V_{i-1}) = 0$ and thus~$I.V_i \subseteq V_{i-1}$ for every~$i = 1, \dotsc, n$.
      Whence~$\rho(x)^n = 0$ for every~$x \in I$.
    \qedhere
  \end{description}
\end{proof}

\begin{corollary}
  \label{srad is nilpotent}
  For every finite dimensional Lie~algebra~$\glie$ the ideal~$\srad(\glie)$ is nilpotent.
\end{corollary}

\begin{proof}
  By applying \cref{characterization of nilpotent radical} to the adjoint action of~$\srad(\glie)$ on~$\glie$ we see that every~$x \in \srad(\glie)$ is~{\nilpotent{$\ad_{\glie}$}} nilpotent and hence~{\nilpotent{$\ad_{\srad(\glie)}$}}.
  It follows from Engel’s theorem that~$\srad(\glie)$ is nilpotent.
\end{proof}





\subsubsection*{Showing that~$\srad(\glie) \subseteq \rad(\glie) \cap [\glie, \glie]$}

\begin{lemma}
  \label{seperating points}
  If~$\glie$ is an abelian Lie~algebra then elements of~$\glie$ can be separated by {\onedimensional} representations, in the sense that there exists for all~$x, y \in \glie$ some {\onedimensional} representation~$(V,\rho)$ of~$\glie$ with~$\rho(x) \neq \rho(y)$.
\end{lemma}

\begin{proof}
  A one dimensional representation of~$\glie$ is (up to isomorphism) the same as a Lie~algebra homomorphism~$\glie \to \gllie(k) = k$.
  Both~$\glie$ and~$k$ are abelian, so every linear map~$\glie \to k$ is already a homomorphism of Lie~algebras.
  The assertian is therefore equivalent to~$\glie^*$ separating the elements of~$\glie$, which is known from linear algebra.
\end{proof}

\begin{corollary}
  \label{abelian have zero srad}
  If~$\glie$ is an abelian Lie~algebra then~$\srad(\glie) = 0$.
\end{corollary}

\begin{proof}
  An element~$x \in \srad(\glie)$ annihilates every finite dimensional irreducible representation of~$\glie$ and thus in particular every {\onedimensional} representation of~$\glie$.
  This means that~$x$ cannot be seperated from~$0$ by {\onedimensional} representations of~$\glie$.
  Therefore~$x = 0$ by \cref{seperating points}.
\end{proof}

\begin{lemma}
  \label{functoriality of srad}
  If~$\phi \colon \glie \to \hlie$ is a surjective homomorphism of Lie~algebras then
  \[
    \phi(\srad(\glie))
    \subseteq
    \srad(\hlie)  \,.
  \]
\end{lemma}

\begin{proof}
  Every finite dimensional irreducible representation~$(V, \rho)$ of~$\hlie$ can be pulled back to a representation~$(V, \rho \circ \phi)$ of~$\glie$.
  It follows from the surjectivity of~$\glie$ that~$(V, \rho \circ \phi)$ is again irreducible.
  We find that~$(V, \rho \circ \phi)$ is annihilated by~$\srad(\glie)$, which means that~$\phi(\srad(\glie))$ annihilates~$(V, \rho)$.
  This shows that~$\phi(\srad(\glie))$ annihilates every finite dimensional irreducible representation of~$\hlie$, so~$\phi(\srad(\glie)) \subseteq \srad(\hlie)$.
\end{proof}

\begin{corollary}
  \label{inclusion srad in intersection}
  If~$\glie$ is a finite dimensional Lie~algebra then~$\srad(\glie) \subseteq \rad(\glie) \cap [\glie, \glie]$.
\end{corollary}

\begin{proof}
  That~$\srad(\glie)$ is contained in~$\rad(\glie)$ follows from~$\srad(\glie)$ being nilpotent and hence solvable.
  If~$\pi \colon \glie \to \glie/[\glie, \glie]$ is the canonical projection then
  \[
    \pi(\srad(\glie))
    \subseteq
    \srad(\glie/[\glie, \glie])
    =
    0
  \]
  by \cref{functoriality of srad} and \cref{abelian have zero srad}, and hence~$\srad(\glie) \subseteq \ker \pi = [\glie, \glie]$.
\end{proof}



\subsubsection*{Showing~$\rad(\glie) \cap [\glie, \glie] \subseteq \srad(\glie)$}

For this inclusion we will need some preparation.
We start with a classical observation from linear algebra:

\begin{lemma}
  \label{trace zero lemma}
  Let~$\ringchar(k) = 0$.
  If~$A \in \mat(n, k)$ is a matrix with~$\tr(A^m) = 0$ for all~$m \geq 1$ then~$A$ is nilpotent.
\end{lemma}

\begin{proof}
  We may assume that~$k$ is algebraically closed.
  Let~$\lambda_1, \dotsc, \lambda_r$ be the pairwise different nonzero (!) eigenvalues of~$A$ with corresponding multiplicities~$n_1, \dotsc, n_r$.
  The matrix~$A$ is triangularizable with diagonal entries~$\lambda_1, \dotsc, \lambda_r, 0$ so we need to show that~$n_1 = \dotsb = n_r = 0$.
  
  We have for all~$m \geq 0$ that
  \[
    0
    =
    \tr(A^m)
    =
    n_1 \lambda_1^m + \dotsb + n_r \lambda_r  \,.
  \]
  For~$m = 1, \dotsc, r$ we can rearrange these equalities in the matrix form
  \begin{equation}
    \label{vandermonde product with vector}
    \begin{pmatrix}
      \lambda_1   & \lambda_2   & \cdots  & \lambda_r   \\
      \lambda_1^2 & \lambda_2^2 & \cdots  & \lambda_r^2 \\
      \vdots      & \vdots      & \ddots  & \vdots      \\
      \lambda_1^r & \lambda_2^r & \cdots  & \lambda_r^r
    \end{pmatrix}
    \cdot
    \begin{pmatrix}
      n_1     \\
      n_2     \\
      \vdots  \\
      n_r
    \end{pmatrix}
    =
    0 \,.
  \end{equation}
  If we denote the matrix on the left hand side by~$V$ then the determinant
  \[
    \det(V)
    =
    \lambda_1 \dotsm \lambda_r \prod_{j > i} (\lambda_j - \lambda_i)
  \]
  does not vanish.
  The matrix~$V$ is hence invertible.
  It follows that the column vector in~\eqref{vandermonde product with vector} is the zero vector, so that~$n_1 = \dotsb = n_r = 0$ (here we use that~$\ringchar(k) = 0$).
\end{proof}

\begin{lemma}
  \label{technical lemma}
  Let~$V$ be a finite dimensional vector space and let~$\glie$ be a Lie~subalgebra of~$\gllie(V)$ such that~$V$ is irreducible as a~{\representation{$\glie$}}.
  If~$I$ is any commutative ideal in~$\glie$ then~$I \cap [\glie, \glie] = 0$.
\end{lemma}

\begin{proof}
  Let~$A$ be the (associative and unital)~{\subalgebra{$k$}} of~$\End_k(V)$ generated by~$I$.
  Note that~$A$ is commutative because~$I$ is abelian.
  
  \begin{claim*}
    If~$J$ is a Lie~ideal in~$\glie$ with~$J \subseteq I$ and~$\tr(xa) = 0$ for all~$x \in J$,~$a \in A$ then~$J = 0$.
  \end{claim*}
  
  \begin{proof}[Proof of the Claim]
    It follows from~$x \in J \subseteq I \subseteq A$ that~$x^n \in A$ for all~$n \geq 0$ and thus by assumption~$\tr(x^n) = \tr(x \cdot x^{n-1}) = 0$ for all~$n \geq 1$.
    It follows from \cref{trace zero lemma} that~$x$ is nilpotent and it therefore further follows from \cref{nilpotent on irrep is zero} that~$J = 0$.
  \end{proof}

  We find that~$\glie$ and~$I$ commute:
  Letting~$x \in \glie$ and~$y \in I$ we find for all~$a \in A$ that~$ay = ya$ because~$A$ is commutative.
  As a consequence,
  \begin{align*}
    \tr([x,y] a)
    &=
    \tr(xya - yxa)
    \\
    &=
    \tr(xya) - \tr(yxa)
    \\
    &=
    \tr(xya) - \tr(xay)
    \\
    &=
    \tr(xya) - \tr(xya)
    \\
    &=
    0 \,.
  \end{align*}
  It follows with the claim that~$[\glie, I] = 0$.
  
  It follows that~$\glie$ and~$A$ commute:
  The centralizer of~$\glie$ in~$\End_k(V)$ is an (associative and unital) subalgebra of~$\End_k(V)$ that contains~$I$, and hence also contains~$A$.
  
  We see that~$\tr(za) = 0$ for all~$z \in [\glie, \glie]$ and~$a \in A$:
  We have for all~$x, y \in \glie$ that~$ay = ya$ because~$\glie$ and~$A$ commute, and hence~$\tr([x,y]a) = 0$ by the same calculation as above.
  
  It follows for~$J = [\glie, \glie] \cap I$, which is a Lie~ideal in~$\glie$ that is contained in~$I$, that~$\tr(za) = 0$ for all~$z \in J$ and~$a \in A$ and thus~$J = 0$ by the claim.
  This is what we want to prove.
\end{proof}

\begin{corollary}
  \label{inclusion intersection in srad}
  If~$\glie$ is a Lie~algebra then~$\rad(\glie) \cap [\glie, \glie] \subseteq \srad(\glie)$.
\end{corollary}

\begin{proof}
  Let~$(V, \rho)$ be a finite dimensional irreducible representation of~$\glie$.
  Then~$\rho(\rad(\glie))$ is solvable and hence there exists some minimal~$n \geq 0$ with~$\rho(\rad(\glie))^{(n+1)} = 0$;
  here we use for every ideal~$I$ the convention~$I^{(0)} = I$, so that~$I^{(m+1)} = [I^{(m)}, I^{(m)}]$ for all~$m \geq 0$.
  Let~$\glie' \defined \rho(\glie)$ and~$I' \defined \rho(\rad(\glie))^{(n)}$.
  Then~$I'$ is an ideal in~$\glie'$ that is abelian because
  \[
    [I', I']
    =
    [\rho(\rad(\glie))^{(n)}, \rho(\rad(\glie))^{(n)}]
    =
    \rho(\rad(\glie))^{(n+1)}
    =
    0 \,.
  \]
  That~$V$ is irreducible as a representation of~$\glie$ means that it is irreducible as a representation of~$\glie'$.
  We can therefore apply \cref{technical lemma} to find that~$I' \cap [\glie', \glie'] = 0$.
  Hence
  \begin{equation}
    \label{rho of intersection}
    0
    =
    I' \cap [\glie', \glie']
    =
    \rho(\rad(\glie))^{(n)} \cap [\rho(\glie), \rho(\glie)]
    =
    \rho( \rad(\glie)^{(n)} \cap [\glie, \glie] ) \,.
  \end{equation}
  If now~$n \geq 1$ then~$\rad(\glie)^{(n)} \subseteq [\glie, \glie]$ and then~\eqref{rho of intersection} simplifies to~$\rho( \rad(\glie)^{(n)} ) = 0$.
  But this would contradict the minimality of~$n$, hence~$n = 0$.
  We hence find
  \[
    0
    =
    \rho( \rad(\glie) \cap [\glie, \glie] ) \,.
  \]
  This shows that~$\rad(\glie) \cap [\glie, \glie]$ annihilates every finite dimensional irreducible representation of~$\glie$, which means that~$\rad(\glie) \cap [\glie, \glie] \subseteq \srad(\glie)$.
\end{proof}

With this we finally arrive at the desired equality:

\begin{theorem}
  If~$\glie$ is a finite dimensional Lie~algebra then~$\srad(\glie) = \rad(\glie) \cap [\glie, \glie]$.
\end{theorem}

\begin{proof}
  We combine \cref{inclusion srad in intersection} and \cref{inclusion intersection in srad}.
\end{proof}





\subsection*{A Generalization}

The nilpotence of~$[\glie, \rad(\glie)]$ can also be generalized by using Problem~20 from sheet~4, as explained in \cite[I.{\S}5.5]{bourbaki_lie} and \cite[\S19.6]{tauvel_yu}:

\begin{theorem}
  \label{derivations map radical into nilradical}
  If~$D$ is a derivation of a finite dimensional Lie~algeba~$\glie$ then
  \[
    D(\rad(\glie))
    \subseteq
    \nil(\glie) \,.
  \]
\end{theorem}

We are now gonna prove this \lcnamecref{derivations map radical into nilradical}.
We have split up the proof in multiple small steps each of which is interesting on its own.

\begin{definition}
  An ideal~$I$ is a Lie~algebra~$\glie$ is \emph{characteristic} if~$D(I) \subseteq I$ for every ideal~$I$.
\end{definition}

\begin{example}
  The derived ideal~$[\glie, \glie]$ is characteristic because
  \[
    D([\glie, \glie])
    \subseteq
    [D(\glie), \glie] + [\glie, D(\glie)]
    \subseteq
    [\glie, \glie]  \,.
  \]
\end{example}

\begin{lemma}
  \label{transitivity of ideals}
  If~$I$ is a characteristic ideal in a Lie~algebra~$\glie$ and~$J$ is an ideal in~$I$ then~$J$ is also an ideal in~$\glie$.
\end{lemma}

\begin{proof}
  For every~$x \in \glie$ the adjoint action~$\ad(x)$ restricts to a derivation of~$I$, which then leaves~$J$ invariant.
\end{proof}

\begin{lemma}
  If~$\glie$ is a finite dimensional Lie~algebra with Killing form~$\kappa$ then every derivation~$D$ of~$\glie$ is skew-selfadjoint with respect to~$\kappa$, i.e. for all~$x, y \in \glie$,
  \[
    \kappa(D(x), y) 
    =
    -\kappa(x, D(y)) \,.
  \]
\end{lemma}

\begin{proof}
  We find with~$[D, \ad(x)] = \ad(D(x))$%
  \footnote{This is the formula that show that the inner derivations form an ideal in~$\Der(\glie)$.}
  that
  \begin{align*}
    \kappa(D(x), y)
    &=
    \tr(\ad(D(x)) \ad(y))
    \\
    &=
    \tr([D, \ad(x)], \ad(y))
    \\
    &=
    \tr(D \ad(x) \ad(y) - \ad(x) D \ad(y))
    \\
    &=
    \tr(D \ad(x) \ad(y)) - \tr(\ad(x) D \ad(y))
    \\
    &=
    \tr(\ad(x) \ad(y) D) - \tr(\ad(x) D \ad(y))
    \\
    &=
    \tr(\ad(x) [\ad(y), D])
    \\
    &=
    -\tr(\ad(x) [D, \ad(y)])
    \\
    &=
    -\tr(\ad(x) \ad(D(y)))
    \\
    &=
    -\kappa(x, D(y))
  \end{align*}
  as claimed.
\end{proof}

\begin{corollary}
  If~$I$ is a characteristic ideal in a finite dimensional Lie~algebra~$\glie$ then its orthogonal~$I^\perp$ with respect to the Killing form~$\kappa$ of~$\glie$ is again a characterestic ideal.
\end{corollary}

\begin{proof}
  The orthogonal of~$I$ is again an ideal by the associativity of~$\kappa$.
  If~$D$ is any derivation of~$\glie$ and~$x \in I^\perp$ then for every~$y \in I$ again~$D(y) \in I$ and hence
  \[
    \kappa(D(x), y)
    =
    -\kappa(x, D(y))
    =
    0 \,.
  \]
  Thus~$D(I^\perp) \subseteq I^\perp$.
\end{proof}

\begin{example}
  \label{radical is characteristic}
  If~$\glie$ is a finite dimensional Lie~algebra then the radical~$\rad(\glie)$ is the orthogonal of the derived ideal~$[\glie, \glie]$ (by part~3) and hence again a characteristic ideal.
\end{example}

\begin{corollary}
  \label{radical of ideal}
  If~$I$ is an ideal in a finite dimensional Lie~algebra~$\glie$ then
  \[
    \rad(I)
    =
    \rad(\glie) \cap I  \,.
  \]
\end{corollary}

\begin{proof}
  The intersection~$\rad(\glie) \cap I$ is a solvable ideal in~$I$ and hence contained in~$\rad(I)$.
  The radical~$\rad(I)$ is a characteristic ideal in~$I$ by \cref{radical is characteristic} and hence an ideal in~$\glie$ by \cref{transitivity of ideals}.
  It is solvable and hence contained in~$\rad(\glie)$, and thus contained in~$\rad(\glie) \cap I$.
\end{proof}


\begin{proof}[Proof of \cref{derivations map radical into nilradical}]
  We apply the construction of Problem~20 from sheet~4:
  We regard the element~$D \in \Der(\glie)$ as a Lie~algebra homomorphism~$\theta \colon k \to \Der(\glie)$ and form the semidirect product~$\glie' \defined \glie \rtimes_\theta k$.
  Then the derivation~$D$ of~$\glie$ extends to the inner derivation~$\ad_{(0,1)}$ of~$\glie'$.
  
  It follows from \cref{radical of ideal} that~$\rad(\glie) = \rad(\glie') \cap \glie$ is contained in~$\rad(\glie')$ because~$\glie$ is an ideal in~$\glie'$.
  Now~$[\glie', \rad(\glie')]$ is contained in the nilradical~$\nil(\glie')$ as seen in \cref{srad is nilpotent}.
  Hence
  \[
    D( \rad(\glie) )
    =
    \ad_{(0,1)}( \rad(\glie) )
    =
    [(0,1), \rad(\glie)]
  \]
  is contained in~$\nil(\glie')$, and thus also in~$\glie \cap \nil(\glie')$.
  This intersection is a nilpotent ideal in~$\glie$ and hence contained in~$\nil(\glie)$.
  Thus~$D(\rad(\glie)) \subseteq \glie \cap \nil(\glie') \subseteq \nil(\glie)$.
\end{proof}
