\section{}

In the following~$k$ denotes an algebraically closed field.
We start by doing all the preparation to effectively solve the exercise:





\subsection*{Preparation}

\begin{definition}
  A \emph{flag} in a vector space~$V$ is an increasing sequence of linear subspaces
  \[
    0
    =
    F_0
    \subsetneq
    F_1
    \subsetneq
    \dotsb
    \subsetneq
    F_n
    =
    V \,.
  \]
  This flag is \emph{complete} if it has maximal length, i.e.\ if~$V$ is finite dimensional and~$\dim F_i = i$ for every~$i = 0, \dotsc, n$.
\end{definition}

\begin{recall}
  \label{conjugation is a lie algebra iso}
  If~$A$ is a~{\algebra{$k$}} and~$a \in A^\times$ is a unit then the conjugation map~$x \mapsto a x a^{-1}$ is an algebra automorphism of~$A$ and hence also a Lie~algebra automorphism.
\end{recall}

\begin{recall}
  \label{recalling solvability via flags}
  If~$V$ is a representation of a Lie~algebra~$\hlie$ and~$F = (F_i)_{i=0}^n$ is a flag of~$V$ then~$\hlie$ \emph{stabilizes}~$F$ if~$\hlie.F_i \subseteq F_i$ for every~$i = 0, \dotsc, n$, i.e.\ if~$F$ consists of subrepresentations.
  
  If~$V$ is a finite dimensional vector space then the following conditions on a Lie~subalgebra~$\hlie$ of~$\gllie(V)$ are equivalent:
  \begin{enumerate}
    \item
      $\hlie$ is solvable.
    \item
      $\hlie$ stabilizes a complete flag of~$V$.
    \item
      There exists a basis of~$\hlie$ with respect to which~$\hlie$ is represented by upper triangular matrices.
  \end{enumerate}
  For a Lie~subalgebra~$\hlie$ of~$\gllie_n(k)$ the following conditions are equivalent:
  \begin{enumerate}
    \item
      $\hlie$ is solvable.
    \item
      $\hlie$ stabilizes a complete flag of~$k^n$.
    \item
      There exists some~$g \in \GL_n(k)$ such that~$g \hlie g^{-1}$ consists of upper triangular matrices.
  \end{enumerate}
  (That~$k$ is algebraically closed is used to go from the solvability of~$\hlie$ to the other two conditions, which are equivalent over any field, and which conversely imply the solvability of~$\hlie$ over any field.)
\end{recall}

We denote by~$\Flags$ the set of all complete flags in~$k^n$ and by
\[
  S
  \defined
  (0, \gen{e_1}, \gen{e_1, e_2}, \dotsc, \gen{e_1, \dotsc, e_n} = k^n)
\]
the \emph{standard flag} of~$k^n$.

\begin{recall}
  \label{recalling group actions}
  Let~$G$ be a group acting on a set~$X$, i.e.\ let~$X$ be a~{\set{$G$}}.
  The orbit of an element~$x \in X$ is the set~$G.x = \{g.x \suchthat g \in G\}$.
  The action of~$G$ on~$X$ is \emph{transitive} if the following equivalent conditions hold:
  \begin{enumerate}
    \item
      There exists for all~$x, y \in X$ some~$g \in G$ with~$g.x = y$.
    \item
      There exists some~$x \in X$ such that there exists for every~$y \in X$ some~$g \in G$ with~$g.x = y$.
    \item
      The orbit of every~$x \in X$ is all of~$X$.
    \item
      There exists some~$x \in X$ whose orbit is all of~$X$.
  \end{enumerate}
  The \emph{stabilizer} of a point~$x \in X$ is the subgroup~$G_x = \{g \in G \suchthat g.x = x\}$ of~$G$.
  The map
  \[
    G/G_x
    \to
    G.x \,,
    \quad
    g G_x
    \mapsto
    g.x
  \]
  is a well-defined isomorphism of~{\sets{$G$}}, where the action of~$G$ on the set of left cosets~$G/G_x$ is given by~$g.g'G_x = g g' G_x$.
\end{recall}

\begin{lemma}
  \label{action on flags}
  The group~$\GL_n(k)$ acts on the set of complete flags~$\Flags$ via
  \[
    g.(F_0, \dotsc, F_n)
    =
    (g.F_0, \dotsc, g.F_n)  \,.
  \]
  This action is transitive and the stabilizer of the standard flag~$S$ is~$\Borel_n(k)$ (the group of upper triangular~{\matrices{$(n \times n)$}}).
\end{lemma}

\begin{proof}
  This is indeed an action of~$\GL_n(k)$ on~$\Flags$.
  
  If~$F \in \Flags$ with~$F = (F_0, \dotsc, F_n)$ then let~$x_1, \dotsc, x_n$ be a basis of~$V$ so that~$x_1, \dotsc, x_i$ is a basis of~$F_i$ for every~$i = 0, \dotsc, n$.
  Then the matrix~$g \in \mat(n, k)$ with columns~$x_1, \dotsc, x_n$ is invertible and~$g.S = F$ because
  \[
    g.\gen{e_1, \dotsc, e_i}
    =
    \gen{g.e_1, \dotsc, g.e_i}
    =
    \gen{x_1, \dotsc, x_i}
    =
    F_i
  \]
  for every~$i = 0, \dotsc, n$.
  This shows that the action is transitive.
  
  If~$g \in \GL_n(k)$ with~$g.S = S$ then let~$x_1, \dotsc, x_n$ be the columns of~$x$.
  That~$g.S = S$ means that~$g.F_i \subseteq F_i$ for all~$i = 0, \dotsc, n$, which means that
  \[
    \gen{x_1, \dotsc, x_i}
    \subseteq
    \gen{e_1, \dotsc, e_i}
  \]
  for all~$i = 0, \dotsc, n$.%
  \footnote{The inclusion~$g.F_i \subseteq F_i$ is by equality of dimensions equivalent to~$g.F_i = F_i$.
  We choose to work with the condition~$g.F_i \subseteq F_i$ so that we don’t need to worry about equality.}
  This happens to be the case precisely when for all~$i = 0, \dotsc, n$ the~{\howmanyth{$i$}} column of~$g$ admits no entries after the~{\howmanyth{$i$}} row, i.e.\ if~$g$ is upper triangular.
  This shows that the stabilizer of the standard flag~$S$ is precisely~$\Borel_n(k)$.
\end{proof}

We denote by~$\blie$ the Lie~subalgebra of~$\sllie_n(k)$ of traceless upper triangular matrices.

\begin{lemma}
  \label{borel invariant subspaces}
  The only~{\subrepresentations{$\blie$}} of~$k^n$ are~$\gen{e_1, \dotsc, e_i}$ for~$i = 0, \dotsc, n$.
\end{lemma}

\begin{proof}
  Let~$U$ be a~{\subrepresentation{$\blie$}} and let~$x \in U$ be a vector with entries~$x_1, \dotsc, x_n$.
  Suppose that~$x_i \neq 0$ but~$x_{i+1} = \dotsb = x_n = 0$.
  Then~$x/x_i \in U$ and so we may assume~$x_i = 1$.
  Let~$N \in \blie$ be the matrix
  \[
    N
    \defined
    \begin{pmatrix}
      0 & 1 &         &   \\
        & 0 & \ddots  &   \\
        &   & \ddots  & 1 \\
        &   &         & 0
    \end{pmatrix}
  \]
  The vectors
  \[
    x
    =
    \vect{x_1 \\ \vdots \\ x_{i-2} \\ x_{i-1} \\ 1 \\ 0 \\ \vdots \\ 0} \,,
    \quad
    N x
    =
    \vect{x_2 \\ \vdots \\ x_{i-1} \\ 1 \\ 0 \\ 0 \\ \vdots \\ 0} \,,
    \quad
    \dotsc \,,
    \quad
    N^{n-2} x
    =
    \vect{x_{i-1} \\ 1 \\ 0 \\ \vdots \\ \vdots \\ \vdots \\ 0}
    \quad
    N^{i-1} x
    =
    \vect{1 \\ 0 \\ \vdots \\ \vdots \\ \vdots \\ \vdots \\ 0}
  \]
  are again contained in~$U$ and they are linearly independent.
  We also see that
  \[
    \gen{x, Nx, N^2 x, \dotsc, N^{i-1} x}
    =
    \gen{e_1, e_2, \dotsc, e_i}
  \]
  and hence~$\gen{e_1, e_2, \dotsc, e_i} \subseteq U$.
  Let now~$i$ be maximal such that there exists some~$x \in U$ with~$x_i \neq 0$.
  Then~$U \subseteq \gen{e_1, \dotsc, e_i}$ by choice of~$i$ but also~$x_{i+1} = \dotsb = x_n = 0$ and hence~$\gen{e_1, \dotsc, e_i} \subseteq U$ by the above observation.
  Thus~$U = \gen{e_1, \dotsc, e_i}$.
\end{proof}

\begin{corollary}
  \label{standard borel fixes only standard flag}
  The standard flag~$S$ is the only complete flag in~$k^n$ stabilized by~$\blie$.
\end{corollary}

\begin{proof}
  The flag~$S$ is stabilized by~$\blie$.
  On the other hand every stabilized flag consists of~{\subrepresentations{$\blie$}} whence the uniqueness follows from \cref{borel invariant subspaces}.
\end{proof}

\begin{lemma}
  \label{fixed flag under conjugation}
  Let~$\hlie$ be a Lie~subalgebra of~$\gllie_n(k)$ and let~$g \in \GL_n(k)$.
  Then~$\hlie$ stabilizes a flag~$F$ of~$k^n$ if and only if~$g \hlie g^{-1}$ stabilizes the flag~$g.F$.
\end{lemma}

\begin{proof}
  Let~$F = (F_0, \dotsc, F_n)$.
  Then
  \begin{align*}
    {}&
    \text{$\hlie$ stabilizes~$F$}
    \\
    \iff{}&
    \text{$\hlie.F_i \subseteq F_i$ for all~$i = 0, \dotsc, n$}
    \\
    \iff{}&
    \text{$\hlie g^{-1} . g F_i \subseteq F_i$ for all~$i = 0, \dotsc, n$}
    \\
    \iff{}&
    \text{$g \hlie g^{-1} . g F_i \subseteq g F_i$ for all~$i = 0, \dotsc, n$}
    \\
    \iff{}&
    \text{$g \hlie g^{-1}$ stabilizes~$g.F$} 
  \end{align*}
  as claimed
\end{proof}

\begin{corollary}
  \label{unique fixed flag of borel}
  If~$g \in \GL_n(k)$ then~$g.S$ is the only complete flag of~$k^n$ stabilized by~$g \blie g^{-1}$.
\end{corollary}

\begin{proof}
  If~$g \blie g^{-1}$ stabilizes a flag~$F$ then we may write~$F = g'.S$ for some~$g' \in \GL_n(k)$ by \cref{action on flags}.
  Then~$\blie = g^{-1} (g \blie g^{-1}) g$ stabilizes~$g^{-1} g'.S$ by \cref{fixed flag under conjugation}.
  It follows from \cref{standard borel fixes only standard flag} that~$g^{-1} g'.S = S$.
  Therefore~$g^{-1} g' \in \Borel_n(k)$ by \cref{action on flags}.
  We can therefore write the group element~$g'$ as~$g' = g h$ with~$h \in \Borel_n(k)$.
  Then
  \[
    F
    =
    g'.S
    =
    gh.S
    =
    g.h.S
    =
    g.S
  \]
  because~$h.S = S$, which is the desired uniqueness.
\end{proof}





\subsection*{The Exercise Itself}

The Lie~subalgebra~$\blie$ of~$\sllie_n(k)$ is solvable.
If~$\blie'$ is any other solvable Lie~subalgebra of~$\sllie_n(k)$ then by \cref{recalling solvability via flags} there exists some~$g \in \GL_n(k)$ such that~$g \blie' g^{-1}$ consists of upper triangular matrices.
The conjugation action~$g (-) g^{-1}$ gives a Lie~algebra automorphism of~$\sllie_n(k)$ by \cref{conjugation is a lie algebra iso} and because the condition~$\tr(x) = 0$ is invariant under conjugation.
Thus~$g \blie' g^{-1}$ is a solvable Lie~subalgebra of~$\blie$.

This shows that~$\blie$ is of maximal dimension among all solvable Lie~subalgebras of~$\sllie_n(k)$ and thus a Borel~subalgebra.

If~$\blie'$ is itself a Borel~subalgebra then~$g \blie' g^{-1}$ is again a Borel~subalgebra of~$\sllie_n(k)$.
It then follows from the solvability of~$\blie$ and maximality of~$g \blie' g^{-1}$ that~$g \blie' g^{-1} = \blie$.
This shows that all Borel~subalgebras of~$\sllie_n(k)$ are~{\conjugated{$\GL_n(k)$}}.

They are also~{\conjugated{$\SL_n(k)$}}:
Suppose that~$\blie' = g \blie g^{-1}$ for some~$g \in \GL_n(k)$.
We may write~$g = hd$ where~$d$ is the invertible diagonal matrix
\[
  d
  =
  \begin{pmatrix}
    \det(g) &   &         &   \\
            & 1 &         &   \\
            &   & \ddots  &   \\
            &   &         & 1
  \end{pmatrix}
  \,.
\]
Then~$\det h = 1$ and
\[
  \blie'
  =
  g \blie g^{-1}
  =
  (hd) \blie (hd)^{-1}
  =
  h d \blie d^{-1} h^{-1}
  =
  h \blie h^{-1}
\]
because~$d \blie d^{-1} = \blie$ (since~$d$ and~$d^{-1}$ are invertible and diagonal).
This shows that~$\blie'$ and~$\blie$ are~{\conjugated{$\SL_n(k)$}}.%
\footnote{Geometrically speaking, suppose that~$F = (F_0, \dotsc, F_n)$ is a complete flag in~$k^n$ and~$x_1, \dotsc, x_n$ is a basis of~$k^n$ such that~$x_1, \dotsc, x_i$ is a basis of~$F_i$ for every~$i = 0, \dotsc, n$.
If~$g \in \GL_n(k)$ is the matrix with columns~$x_1, \dotsc, x_n$ then it may happen that~$\det(g) \neq 1$.
But we can normalize the first basis vector~$x_1$ by replacing it with~$x_1/\deg(g)$.
The resulting basis~$y_1, \dotsc, y_n$ of~$k^n$ with~$y_1 = x_1/\det(g)$ and~$y_i = x_i$ for~$i = 2, \dotsc, n$ then again satisfies~$\gen{y_1, \dotsc, y_i} = F_i$ for all~$i = 1, \dotsc, n$, but the resulting matrix~$h$ with columns~$y_1, \dotsc, y_n$ now satisfies~$\det(h) = \deg(g)/\det(g) = 1$.
We can therefore adjust the determinant of the occuring matrix~$g$ without changing the underlying flag.}

Let~$\Borels$ be the set of all Borel~subalgebras of~$\sllie_n(k)$.
We now construct a bijection
\[
  \Phi
  \colon
  \Borels
  \xlongto{\sim}
  \Flags \,.
\]
The map~$\Phi$ assigns to each Borel~subalgebra~$\blie'$ of~$\sllie_n(k)$ the unique complete flag of~$k^n$ that~$\blie'$ stabilizes.

This gives a well-defined map:
Every Borel Lie~subalgebra~$\blie'$ of~$\sllie_n(k)$ is of the form~$\blie' = g \blie g^{-1}$ for some~$g \in \SL_n(k)$, hence stabilizes a unique complete flag by \cref{unique fixed flag of borel}, namely~$g.S$ by \cref{fixed flag under conjugation}.
Hence~$\Phi(g \blie g^{-1}) = g.S$ is well-defined.

The map~$\Phi$ is surjective:
If~$F$ is any complete flag in~$k^n$ then~$F = g.S$ for some~$g \in \GL_n(k)$ by \cref{action on flags} and then~$g \blie g^{-1}$ is a Borel~subalgebra of~$\sllie_n(k)$ that is mapped to~$F$.

The map~$\Phi$ is also injective:
Let~$\blie'$ be any Borel~subalgebra of~$\sllie_n(k)$ and let~$F = \Phi(\blie')$ be the complete flag stabilized by~$\blie'$.
Then
\[
  \blie_F
  =
  \{
    x \in \sllie_n(k)
  \suchthat
    \text{$x$ stabilizes~$F$}
  \}
\]
is a Lie~subalgebra of~$\sllie_n(k)$, and we find by \cref{recalling solvability via flags} that~$\blie_F$ is solvable.
The Borel~subalgebra~$\blie'$ is contained in~$\blie_F$ and thus~$\blie' = \blie_F$ by the maximality of~$\blie'$.
This shows that~$\blie'$ is uniquely determined by~$F$.

\begin{remark}
  \leavevmode
  \begin{enumerate}
    \item
      The transitive action of~$\GL_n(k)$ on~$\Flags$ from \cref{action on flags} restricts to an action of~$\SL_n(k)$ on~$\Flags$, which is again transitive:
      We have seen above that any~$g \in \GL_n(k)$ can be written as~$g = hd$ for some~$h \in \SL_n(k)$ and~$d \in \Borel_n(k)$.
      Then
      \[
        g.S
        =
        hd.S 
        =
        h.d.S 
        =
        h.S
      \]
      since~$d.S = S$ (because~$d$ is upper triangular).
      The~{\orbit{$\SL_n(k)$}} of~$S$ thus coincides with the~{\orbit{$\GL_n(k)$}} of~$S$, which is all of~$\Flags$.
      The stabilizer of~$S$ in~$\SL_n(k)$ is~$\SB_n(k) \defined \Borel_n(k) \cap \SL_n(k)$.%
      \footnote{The notation~$\SB_n(k)$ is (probably) non-standard and the author’s fault.}
      It follows from \cref{recalling group actions} that we get an induced bijection
      \begin{alignat*}{2}
        \SL_n(k)/{\SB_n(k)}
        &\to
        \Flags  \,,
        &
        \quad
        g \SB_n(k)
        &\mapsto
        g.S \,.
      \intertext{The group~$\SL_n(k)$ acts (transitively) on~$\Borels$ via conjugation and the subgroup~$\SB_n(k)$ stabilizes the standard flag~$\blie$.
      We therefore get an induced (surjective) map}
        \SL_n(k)/{\SB_n(k)}
        &\to
        \Borels \,,
        &
        \quad
        g \SB_n(k)
        &\mapsto
        g \blie g^{-1} \,.
      \end{alignat*}
      We get overall the following commutative diagram:
      \[
        \begin{tikzcd}[column sep = small, row sep = huge]
          {}
          &
          \SL_n(k)/{\SB_n(k)}
          \arrow{dl}[above left]{g \mapsto g \blie g^{-1}}
          \arrow{dr}[above right]{g \mapsto g.S}
          &
          {}
          \\
          \Borels
          \arrow{rr}[above]{\Phi}[below]{g \blie g^{-1} \mapsto g.S}
          &
          {}
          &
          \Flags
        \end{tikzcd}
      \]
      The bottom map and right map are bijections, so the same holds for the left map.
      We have thus constructed a commutative triangle of one-to-one correspondences.
    \item
      Instead of~$\sllie_n(k)$ and~$\SL_n(k)$ we could have used~$\gllie_n(k)$ and~$\GL_n(k)$.
      We then get another commutative triangle of one-to-one correspondences:
      \[
        \begin{tikzcd}[column sep = small, row sep = huge]
          {}
          &
          \GL_n(k)/{\Borel_n(k)}
          \arrow{dl}[above left]{g \mapsto g \blie g^{-1}}
          \arrow{dr}[above right]{g \mapsto g.S}
          &
          {}
          \\
          \left\{
            \begin{tabular}{@{}c@{}}
              Borel~subalgebras \\
              of~$\gllie_n(k)$
            \end{tabular}
          \right\}
          \arrow{rr}[below]{g \blie g^{-1} \mapsto g.S}
          &
          {}
          &
          \left\{
            \begin{tabular}{@{}c@{}}
              complete flags \\
              in~$k^n$
            \end{tabular}
          \right\}
        \end{tikzcd}
      \]
  \end{enumerate}
  The set~$\Flags$ of complete flags can be regarded as a geometric object, a so called \emph{flag variety}.
  The above diagrams then give us connections between linear algebra groups, Lie~theory and (algebraic) geometry.
\end{remark}








